\section{Physics of SU2}

\begin{frame}{Conservative Equations}
\begin{equation}
\frac{\partial \boldsymbol{U}}{\partial t}+\nabla \cdot \boldsymbol{F}^{c}-\nabla \cdot\left(\mu^{v k} \boldsymbol{F}^{v k}\right)=\boldsymbol{Q} \quad \text { in }\Omega,  \quad t>0
\end{equation}
    
\end{frame}

\subsection{Fluid Model}
\begin{frame}[allowframebreaks]
\frametitle{Reynolds-Averaged Navier–Stokes Equations (RANS)}
\begin{equation}
\left\{\begin{array}{ll}{\mathcal{R}(U)=\frac{\partial U}{\partial t}+\nabla \cdot \boldsymbol{F}^{c}-\nabla \cdot\left(\mu^{v k} \boldsymbol{F}^{v k}\right)-\boldsymbol{Q}=0} & {\text { in } \Omega, \quad t>0} \\ {v=\mathbf{0}} & {\text { on }S,} \\ {\partial_{n} T=0} & {\text { on } S} \\ {(\boldsymbol{W})_{+}=W_{\infty}} & {\text { on } \Gamma_{\infty}}\end{array}\right.
\end{equation}


\begin{equation*}
\frac{\partial \boldsymbol{U}}{\partial t}+\nabla \cdot \boldsymbol{F}^{c}-\nabla \cdot\left(\mu^{v k} \boldsymbol{F}^{v k}\right)=Q \quad \text { in } \Omega, t>0
\end{equation*}

\framebreak

\begin{equation*}
\boldsymbol{U}=\left\{\begin{array}{c}{\rho } \\ {\rho \boldsymbol{v} } \\ {\rho E}\end{array}\right\},\quad
\boldsymbol{F}^{c}=\left\{\begin{array}{c}{\rho \boldsymbol{v}} \\ {\rho \boldsymbol{v} \otimes \boldsymbol{v}+ p\mathcal{I} } \\ {\rho E \boldsymbol{v}+p \boldsymbol{v}}\end{array}\right\}
\end{equation*}

\begin{equation*}
\boldsymbol{F}^{v1}=\left\{\begin{array}{c}{\cdot} \\ {\mathcal{\tau}} \\ {\mathcal{\tau} \cdot \boldsymbol{v}}\end{array}\right\},\quad
\boldsymbol{F}^{v2}=\left\{\begin{array}{c}{\cdot} \\ {\cdot} \\ {c_p \nabla T}\end{array}\right\},\quad
Q=\left\{\begin{array}{c}{q_\rho} \\ {\boldsymbol{q}_{\rho\boldsymbol{v}}} \\ {q_{\rho E}}\end{array}\right\}
\end{equation*}

\begin{equation}
\mathcal{\tau}=\nabla \boldsymbol{v}+\nabla \boldsymbol{v}^{T}-\frac{2}{3} \mathcal{I}(\nabla \cdot \boldsymbol{v})
\end{equation}

For \textbf{ideal gas},
\begin{equation}
    p=(\gamma-1)\rho[E-\frac{1}{2}(\boldsymbol{v}\cdot \boldsymbol{v})],
\end{equation}
\begin{equation}
    T=p/(\rho R), 
\end{equation}
\begin{equation}
    c_p = \gamma R/(\gamma-1)
\end{equation}

The total viscosity is divided into a laminar $\mu_{dyn}$  and a
turbulent $\mu_{tur}$ component.
\begin{equation}
\mu^{v 1}=\mu_{\mathrm{dyn}}+\mu_{\mathrm{tur}}, \quad \mu^{v 2}=\frac{\mu_{\mathrm{dyn}}}{P r_{d}}+\frac{\mu_{\mathrm{tur}}}{P r_{t}},
\end{equation}
where $Pr_d$ and $Pr_t$ are the dynamic and turbulent Prandtl numbers,
respectively.



\framebreak
\textbf{Supported Boundary Conditions}
\begin{itemize}
    \item Euler (flow tangency) and symmetry wall,
    \item no-slip wall (adiabatic and isothermal),
    \item far-field and near-field
boundaries,
    \item characteristic-based inlet boundaries (stagnation, mass
flow, or supersonic conditions prescribed),
    \item characteristic-based
outlet boundaries (back pressure prescribed),
    \item periodic boundaries,
    \item nacelle inflow boundaries (fan face Mach number prescribed),
    \item nacelle exhaust boundaries (total nozzle temp and total nozzle
pressure prescribed).
\end{itemize}

The turbulent viscosity $\mu_{tur}$ is obtained from a suitable turbulence
model involving the flow state and a set of new variables. The (1) \textbf{Menter
shear-stress transport} model and the (2) \textbf{Spalart–Allmaras} model are two
of the most common and widely used turbulence models for the
analysis and design of engineering applications in turbulent flows.
    

\end{frame}

\begin{frame}{SU2 Framework}
\uncover<+->{\begin{equation*}
\frac{\partial U}{\partial t}+\nabla \cdot \boldsymbol{F}^{c}-\nabla \cdot\left(\mu^{v k} \boldsymbol{F}^{v k}\right)=Q \quad \text { in } \Omega, t>0
\end{equation*}
}
\uncover<+->{\[ \Downarrow \]
\begin{align*}
    \int_{\Omega_{i}} \frac{\partial U}{\partial t} d \Omega & + \sum_{j \in \mathcal{N}(i)}\left(\tilde{F}_{i j}^{c}+\tilde{F}_{i j}^{v k}\right) \Delta S_{i j}-Q\left|\Omega_{i}\right| \\ 
=\int_{\Omega_{i}} \frac{\partial U}{\partial t} d \Omega & +  R_{i}(U)=0
\end{align*}
}

For each variable in each cell, there is a residual.
\end{frame}

\begin{frame}{Two-Temperature Model for High-Speed Non-Equilibrium Flow}
    
\end{frame}

\begin{frame}{Heat Equation}
    
\end{frame}

\begin{frame}{Wave Equation}
    
\end{frame}

\begin{frame}{Poisson Equation}
    
\end{frame}

\begin{frame}{Equations of Linear Elasticity}
    
\end{frame}