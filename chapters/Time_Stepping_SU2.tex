\section{BACKUP: Time Integration Method in SU2}

\subsection{Steady Simulations}


\begin{frame}[allowframebreaks]{Implicit Method, Euler Backward Method}

\begin{equation}
\int_{\Omega_{i}} \frac{\partial \boldsymbol{U}_i}{\partial t} d \Omega+\boldsymbol{R}_{i}(\boldsymbol{U}) \approx\left|\Omega_{i}\right| \frac{\mathrm{d} \boldsymbol{U}_{i}}{\mathrm{d} t}+R_{k}(\boldsymbol{U})=0 \quad \rightarrow \quad \frac{\left|\Omega_{i}^{n}\right|}{\Delta t_{i}^{n}} \Delta \boldsymbol{U}_{i}^{n}=-\boldsymbol{R}_{i}\left(\boldsymbol{U}^{n+1}\right)
\end{equation}
where $\Delta \boldsymbol{U}_{i}^{n}=\boldsymbol{U}_{i}^{n+1}-\boldsymbol{U}_{i}^{n}$ means the variables in the cell $i$, $\boldsymbol{R}_i$ is residuals for those variables. However, the residuals at time $n + 1$ are unknown, and linearization about $t^n$ is
needed:

\begin{multline}
    \boldsymbol{R}_{i}\left(\boldsymbol{U}^{n+1}\right)=\boldsymbol{R}_{i}\left(\boldsymbol{U}^{n}\right)+\frac{\partial \boldsymbol{R}_{i}\left(\boldsymbol{U}^{n}\right)}{\partial t} \Delta t_{i}^{n}+\mathcal{O}\left(\Delta t^{2}\right)\\=\boldsymbol{R}_{i}\left(\boldsymbol{U}^{n}\right)+\sum_{j \in \mathcal{N}(i)} \frac{\partial \boldsymbol{R}_{i}\left(\boldsymbol{U}^{n}\right)}{\partial \boldsymbol{U}_{j}} \Delta \boldsymbol{U}_{j}^{n}+\mathcal{O}\left(\Delta t^{2}\right)
\end{multline}

\begin{equation}
\left(\frac{\left|\Omega_{i}\right|}{\Delta t_{i}^{n}} \delta_{i j}+\frac{\partial \boldsymbol{R}_{i}\left(\boldsymbol{U}^{n}\right)}{\partial \boldsymbol{U}_{j}}\right) \cdot \Delta \boldsymbol{U}_{j}^{n}=-\boldsymbol{R}_{i}\left(\boldsymbol{U}^{n}\right),
\end{equation}

If a flux $\tilde{F}_{i j}$ has a stencil of points
${i,j}$, then contributions are made to the Jacobian at four points:
\begin{equation}
\frac{\partial \boldsymbol{R}}{\partial \boldsymbol{U}} :=\frac{\partial \boldsymbol{R}}{\partial \boldsymbol{U}}+\left[\begin{array}{ccccc}{\ddots} & {} & {} & {} & {} \\ {} & {\frac{\partial \tilde{F}_{i j}}{\partial U_{i}}} & {\cdots} & {\frac{\partial \tilde{F}_{i j}}{\partial U_{j}}} \\ {} & {\vdots} & {\ddots} & {\vdots} \\ {} & {-\frac{\partial \tilde{F}_{i j}}{\partial U_{i}}} & {\cdots} & {-\frac{\partial \tilde{F}_{i j}}{\partial U_{j}}} \\ {} & {} & {} & {\ddots}\end{array}\right]
\end{equation}
The position of these four terms are to be checked.


The matrix is very large and contains $(\text{Cell Num} * \text{Variable Num})^2$ components. If one doesn't use the implicit methods in the time-stepping procedure, then there is no necessity to solve the huge sparse linear algebra problem subtly, because all components of the matrix are allocated at the diagonal line.  

Note that, despite implicit schemes being unconditionally stable in theory, a specific value of $\Delta t_i^n$ is needed
to relax the problem. SU2 uses a local-time-stepping technique to accelerate convergence to a steady state.
Local-time-stepping allows each cell in the mesh to advance at a different local time step. Calculation of the
local time step requires the estimation of the eigenvalues and first-order approximations to the Jacobians at
every node i according to

\begin{equation}
\Delta t_{i}=N_{C F L} \min \left(\frac{\left|\Omega_{i}\right|}{\lambda_{i}^{\operatorname{conv}}}, \frac{\left|\Omega_{i}\right|}{\lambda_{i}^{v i s c}}\right)
\end{equation}

where $N_{CFL}$ is the Courant-Friedrichs-Lewy (CFL) number, $\left|\Omega_{i}\right|$ is the volume of the cell $i$ and $\lambda_{i}^{v i s c}$
is the integrated convective spectral radius computed as

\begin{equation}
\lambda_{i}^{c o n v}=\sum_{j \in \mathcal{N}(i)}\left(\left|\vec{u}_{i j} \cdot \vec{n}_{i j}\right|+c_{i j}\right) \Delta S
\end{equation}

where $\vec{u}_{i j}=\left(\vec{u}_{i}+\vec{u}_{j}\right) / 2$, and $c_{i j}=\left(c_{i}+c_{j}\right) / 2$ denote the velocity and the speed of sound at the cell face.
$\vec{n}_{i j}$ denotes the normal direction of the control surface and $\Delta S$, its area. On the other hand, the viscous
spectral radius $\lambda_{i}^{v i s c}$ is computed as
\begin{equation}
\lambda_{i}^{v i s c}=\sum_{j \in \mathcal{N}(i)} C \frac{\mu_{i j}}{\rho_{i j}} S_{i j}^{2}
\end{equation}
where $C$ is a constant, $\mu_{i j}$ is the sum of the laminar and eddy viscosities in a turbulent calculation and $\rho_{i j}$
is the density evaluated at the midpoint of the edge ${i,j}$.

\end{frame}

\begin{frame}[allowframebreaks]{Dual-Time-Stepping}
\begin{equation}
\frac{\partial \boldsymbol{U}}{\partial \tau}+\boldsymbol{R}_{i}^{*}(\boldsymbol{U})=0
\end{equation}
with
\begin{equation}
\begin{aligned} R_{i}^{*}(\boldsymbol{U})=& \frac{3}{2 \Delta t} \boldsymbol{U}_{i} \\+& \frac{1}{\left|\Omega_{i}\right|^{n+1}}\left(\boldsymbol{R}_{i}(\boldsymbol{U})-\frac{2}{\Delta t}\left|\Omega_{i}\right|^{n} \boldsymbol{U}_{i}^{n}+\frac{1}{2 \Delta t}\left|\Omega_{i}\right|^{n-1} \boldsymbol{U}_{i}^{n-1}\right) \end{aligned}
\end{equation}
for second-order accuracy in time (backward difference formula),
where $\Delta t$ is the physical time step, $\tau$ is a fictitious time used to
converge the steady-state problem, $\boldsymbol{R}_i(\boldsymbol{U})$ denotes the residual of the
governing equations.

\end{frame}