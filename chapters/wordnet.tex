%!TEX root = Intro-NLP-seminar.tex
\section{WordNets: lexical semantic networks}

\newcommand{\mysynset}[1]{\text{(#1)}}
\newcommand{\myrel}[1]{\text{\emph{\color{purple}#1}}}
\newcommand{\myword}[1]{\text{`#1'}}
\newcommand{\myconcept}[1]{\text{\scshape #1}}
\newcommand{\myinst}[1]{\text{\scshape\color{green!30!black} #1}}

\begin{frame}
\frametitle{WordNet: An Electronic Lexical Database}

\begin{itemize}
\item \parencite{miller:wordnet:1995}
\item Developed by Princeton University Cognitive Science Laboratory for (American) English
\item Lexical entries organised by \emph{meaning} (semantic content)
\item Wordnets for many other languages have been developed
%\item For humans: new ways of navigating a dictionary/lexicon to learn about words
%\item For computers: resource for performing (some level of) semantic analysis on natural language text
\end{itemize}
\end{frame}

\subsection{Synsets}
\begin{frame}
\frametitle{Synsets = ``synonym set''}
\begin{itemize}
\item Basic unit; represents a word meaning by \alert{synonyms}, \alert{gloss} and \alert{relations to other synsets}
\item Different senses of a word (collocation, phrasal verb, etc.) are placed in different synsets according to parts-of-speech
\item Each synset contains senses of different words that are considered synonymous
%\item For each synset:
%  \begin{itemize}
%    \item gloss
%    \item semantic field
%    \item example usage(s)
%    \item sentence frame (verbs only)
%  \end{itemize}
\end{itemize}
\end{frame}

\begin{frame}[fragile]
\frametitle{First 3 senses for noun ``court''}

\begin{itemize}
	\item 3 synsets, one for each sense (meaning)
	\item Each synset contain member lemmas with same meaning, same POS
	\item Each synset has a definition text; may have example sentence
\end{itemize}

\bigskip

\lstset{keywordstyle=\color{red},keywords={court}}
\begin{lstlisting}
<noun.group> court, tribunal, judicature - (an assembly (including one or more judges) to conduct judicial business)

<noun.group> court, royal_court - (the sovereign and his advisers who are the governing power of a state)

<noun.artifact> court - (a specially marked horizontal area within which a game is played; "players had to reserve a court in advance")
\end{lstlisting}

\end{frame}


\begin{frame}
\frametitle{Synset Organisation}
    
\begin{itemize}
\item 4 synset categories

\bigskip

\begin{center}
\begin{tabular}{l >{\ttfamily}c c}
\toprule
Category & POS code & numerical prefix\\
\midrule
noun & n & 1 \\
verb & v & 2 \\
adjective & a, s & 3 \\
adverb & r & 4 \\
\bottomrule
\end{tabular}
\end{center}

\bigskip

\item Primary key: 9-digit synset ID \alert{or} POS code + 8-digit synset ID
\item WN3.0 synset (court, tribunal, judicature) can be identified by \alert{108329453} or \alert{n-08329453} or \alert{08329453-n} in different systems
\end{itemize}
\end{frame}

\begin{frame}
\frametitle{Mapping Stanford Parser POS codes to WN POS code}

\begin{algorithmic}
\STATE \COMMENT{Ignore all other POS when looking up WN}
\IF{\texttt{\$stanfordPOS} starts with `N'}
	\STATE \texttt{\$wnPOS} $\leftarrow$ `n'
	\STATE \texttt{\$wnPOSnum} $\leftarrow$ 1

\ELSIF{\texttt{\$stanfordPOS} starts with `V'}
	\STATE \texttt{\$wnPOS} $\leftarrow$ `v'
	\STATE \texttt{\$wnPOSnum} $\leftarrow$ 2

\ELSIF{\texttt{\$stanfordPOS} starts with `J'}
	\STATE \texttt{\$wnPOS} $\leftarrow$ \{`a', `s'\}
	\STATE \texttt{\$wnPOSnum} $\leftarrow$ 3

\ELSIF{\texttt{\$stanfordPOS} starts with `R'}
	\STATE \texttt{\$wnPOS} $\leftarrow$ `r'
	\STATE \texttt{\$wnPOSnum} $\leftarrow$ 4
\ENDIF 
\end{algorithmic}


\end{frame}


\subsection{Relations}
\begin{frame}
\frametitle{Noun Synset Relations}
%\framesubtitle{Nouns}
%WordNet specifies relations between synsets and (compound) words

%\begin{exampleblock}{Noun Synset Relations}
\mode<presentation>{\small}
\begin{description}
\item[hypernymy] \mysynset{court, royal court} \myrel{is-a-kind-of} \mysynset{government, authorities, regime}
\item[holonymy] \mysynset{finger} \myrel{is-part-of} \mysynset{hand, manus, mitt, paw} \\
                \mysynset{flour} \myrel{is-substance-of} \mysynset{bread}, \mysynset{dough}, \mysynset{pastry} \\
                \mysynset{jury} \myrel{is-member-of} \mysynset{court, tribunal, judicature}
\item[instance] \mysynset{Mozart, Wolfgang Amadeus Mozart} \myrel{is-instance-of} \mysynset{composer}
\end{description}
(\ldots{}and their respective inverse relations)
%\end{exampleblock}
\end{frame}

\begin{frame}
\frametitle{Verb Synset Relations}
%\framesubtitle{Verbs}
%WordNet specifies relations between synsets and (compound) words

%\begin{exampleblock}{Verb Synset Relations}
\mode<presentation>{\small}
\begin{description}
\item[hypernymy] \mysynset{stroll, saunter} \myrel{is-one-way-to} \mysynset{walk}
\item[troponymy] \mysynset{fear} \myrel{has-specific-way} \mysynset{panic}
\item[cause] \mysynset{teach} \myrel{causes} \mysynset{learn, larn, acquire}
\item[entailment] \mysynset{buy, purchase} \myrel{entails} \mysynset{pay}, \mysynset{choose, take, select, pick out}
\item[verb frames] \mysynset{attack, assail}: \\\texttt{Somebody --s something}\\\texttt{Somebody --s somebody}\\(very simple, without any extra info)
\end{description}
%\end{exampleblock}
\end{frame}

\begin{frame}
\frametitle{Lexical Relations}
\mode<presentation>{\small}
\begin{description}
\item[antonymy] \myword{ugliness} \myrel{$\times$} \myword{beauty}, 
                \myword{pull} \myrel{$\times$} \myword{push},\\
                \myword{difficult} \myrel{$\times$} \myword{easy}, 
                \myword{quickly} \myrel{$\times$} \myword{slowly}
\item[attribute] \myword{strength} \myrel{has-attributes}: \myword{delicate}, \myword{rugged}, \myword{weak}, \myword{strong}
\item[derivation] \myword{maintain} \myrel{is-derivationally-related-to} \myword{maintainable}, \myword{maintenance}, \myword{maintainer}
\item[domain] \myword{medicine} \myrel{topic-has-terms}: \myword{acute}, \myword{fulgurating}, \myword{gauze}, \ldots\\
              \myword{France} \myrel{region-has-terms}: \myword{Battle of Valmy}, \myword{Bastille}, \myword{jeu d'esprit}, \ldots\\
              \myword{colloquialism} \myrel{usage-has-terms}: \myword{lousy}, \myword{humongous}, \myword{gobsmacked}, \ldots
\item[pertainym] \myword{biannual} \myrel{pertains-to} \myword{year}, 
                 \myword{ancestral} \myrel{pertains-to} \myword{ancestor},
                 \myword{Liverpudlian} \myrel{pertains-to} \myword{Liverpool}
\item[participle] \myrel{a} \myword{handheld} \myrel{something-participates-in} \myword{hold}
\end{description}
\end{frame}

\subsection{Getting the Data}

\begin{frame}
\frametitle{Princeton WordNet Data}
    
\begin{itemize}[<+->]
\item Browse/explore online: \url{http://wordnet.princeton.edu/}
\item Searching WordNet: APIs for many programming languages available
\item \ldots But I recommend downloading WordNet as MySQL data
\item Then use whatever programming language you like to query
\end{itemize}

\end{frame}


\begin{frame}[fragile]
\frametitle{WordNet SQL}
    
\begin{itemize}
\item \url{http://wnsql.sourceforge.net/}
\item \menu{Download > wnsql > mysql > 3.0 > mysql-3.0.0-30-wn-30.zip} Unzip. 
\item Create a MySQL database, e.g.~ \texttt{wordnet30}
\item Start a Windows command prompt.\\\menu{Windows Start > Run} type \menu{cmd} \return

\begin{lstlisting}[backgroundcolor=\color{black},basicstyle=\ttfamily\color{green}, 
escapechar=|, framexleftmargin=1em, xleftmargin=1em]
> cd <folder containing unzipped contents> |\return|
> restore.bat |\return|
\end{lstlisting}

\item You'll be prompted for the database name, username and password. Wait while the data is copied into tables.

\item (May need to add \path{C:\xampp\mysql\bin} to system path)
\end{itemize}

\end{frame}



\begin{frame}[fragile]
\frametitle{Some Tables in WN-MySQL}
    
\begin{tikzpicture}[every node/.style={rectangle split, rectangle split parts=2, draw},
every text node part/.style={align=center}] 
\node (words) [text width=4em] 
{words \nodepart{two}{\textbf{wordid}\\lemma}};

\node (senses) [text width=5em, right=3em of words]
{senses \nodepart{two}\textbf{senseid}\\wordid\\synsetid\\sensenum\\\ldots};

\node (synsets) [text width=5em, right=3em of senses]
{synsets \nodepart{two}\textbf{synsetid}\\pos\\definition\\\ldots};

\path[-crow's foot] (words) edge (senses);
\path[-crow's foot] (synsets) edge (senses);

\pause

\node (semlinks) [text width=5em, right = 4em of synsets]
{semlinks \nodepart{two}synset1id\\synset2id\\linkid};

\node (linktypes) [text width=5em, below = 3em of semlinks]
{linktypes \nodepart{two}\textbf{linkid}\\link\\\ldots};

\path[crow's foot-crow's foot] (semlinks) edge (synsets);
\path[-crow's foot] (linktypes) edge (semlinks);
\end{tikzpicture}

\end{frame}

\begin{frame}[fragile]
\frametitle{Querying Synsets (Senses) of a Lemma}
    
\begin{lstlisting}[language=SQL]
SELECT lemma, synsetid, definition
FROM words INNER JOIN senses USING (wordid)
     INNER JOIN synsets USING (synsetid)
WHERE lemma = 'plant' AND pos = 'n';
\end{lstlisting}

\begin{lstlisting}[basicstyle=\ttfamily\scriptsize,columns=fixed]
+-------+-----------+--------------------------------------------+
| lemma | synsetid  | definition                                 |
+-------+-----------+--------------------------------------------+
| plant | 100017222 | (botany) a living organism ....            |
| plant | 103956922 | buildings for carrying on industrial labor |
| plant | 105906080 | something planted secretly for...          |
| plant | 110438470 | an actor situated in the audience          |
+-------+-----------+--------------------------------------------+
4 rows in set (0.00 sec)
\end{lstlisting}

\end{frame}


\begin{frame}[fragile]
\frametitle{Querying Synonyms of a Particular Sense (Synset)}
    
\begin{lstlisting}[language=SQL]
SELECT lemma
FROM words INNER JOIN senses USING (wordid)
     INNER JOIN synsets USING (synsetid)
WHERE synsetid = 103956922;
\end{lstlisting}

\begin{lstlisting}
+------------------+
| lemma            |
+------------------+
| industrial plant |
| plant            |
| works            |
+------------------+
3 rows in set (0.00 sec)
\end{lstlisting}

\end{frame}


\begin{frame}[fragile]
\frametitle{Looking up Semantic Relations}

\begin{lstlisting}[language=SQL]
SELECT synset2id
FROM semlinks INNER JOIN synsets A 
                 ON (A.synsetid = semlinks.synset1id)
              INNER JOIN linktypes USING (linkid)
WHERE A.synsetid = 103956922 AND LINK = 'hypernym';

SELECT lemma
FROM words INNER JOIN senses USING (wordid)
     INNER JOIN synsets USING (synsetid)
WHERE synsetid = 102914991;
\end{lstlisting}

\begin{columns}[T]
\begin{column}{.45\textwidth}
\begin{lstlisting}[basicstyle=\ttfamily\footnotesize]
+-----------+
| synset2id |
+-----------+
| 102914991 |
+-----------+
1 row in set (0.00 sec)
\end{lstlisting}
\end{column}

\begin{column}{.49\textwidth}
\begin{lstlisting}[basicstyle=\ttfamily\footnotesize]
+------------------+
| lemma            |
+------------------+
| building complex |
| complex          |
+------------------+
2 rows in set (0.00 sec)
\end{lstlisting}
\end{column}
\end{columns}

\end{frame}


\subsection{Other Lexical Resources Linking to WordNet}

\begin{frame}
\frametitle{Multilingual Wordnets}
    
\begin{itemize}[<+->]
\item Wordnets in different languages -- same architecture 
\item Some free, some not: \url{http://globalwordnet.org/}
\item Almost all are `linked' to PWN (English) by \alert{synsetid}
\item WordNet Bahasa (\url{http://wn-msa.sourceforge.net/}) \parencite{Bond:etal:WordNetBahasa:2014}
\item More languages: Open Multilingual WordNet (\url{http://compling.hss.ntu.edu.sg/omw/}) \parencite{Bond:Paik:2012}
\end{itemize}

\end{frame}

\begin{frame}
\frametitle{Looking up Synsets in Other WordNets}
%\begin{itemize}
%\item Refer to relations between English WordNet synsets
%\item Copy selected relations over to Malay WordNet where possible
%\end{itemize}

%\includegraphics[width=\textwidth]{synsets-svg}
\begin{center}

\tikzstyle{synset}=[text width=15mm, font=\footnotesize,text centered]
\tikzstyle{arrlabel}=[<-,above,font=\scriptsize]
\tikzstyle{corr}=[<->,dashed]
\tikzstyle{every edge}=[draw,>=stealth']
\begin{tikzpicture}
\path[node distance=32mm] node[synset] (dictionary) {(dictionary, lexicon)}
      node[synset,right of=dictionary] (wordbook) {(wordbook)}
      edge[arrlabel] node {hyper} (dictionary)
      node[synset,right of=wordbook,text width=22mm] (reference) {(reference,\\reference book)}
      edge[arrlabel] node {hyper} (wordbook)
      node[synset,right of=reference,text width=] (book) {(book)}
      edge[arrlabel] node {hyper} (reference);

\node[above of=dictionary,xshift=45mm,yshift=10mm,font=\bfseries] {English};

\begin{scope}[font=\footnotesize\color{mLightBrown}]
\node[above of=dictionary] {106418901};
\node[above of=wordbook] {106418693};
\node[above of=reference] {106417598};
\node[above of=book] {106410904};
\end{scope}

\begin{scope}[node distance=20mm]
\uncover<2->{
\path node[synset,below of=dictionary] (kamus) {(leksikon, kamus)}
      edge[corr] (dictionary)
      node[synset,below of=reference] (rujukan) {(rujukan)}
      edge[corr] (reference)
      node[synset,below of=book,text width=] (buku) {(buku)}
      edge[corr] (book);
\node[below of=dictionary,yshift=-15mm,xshift=45mm,font=\bfseries] {Malay};
}
\uncover<3->{
\path (rujukan) edge[arrlabel] node{hyper} (kamus);
}
\uncover<4->{
\path (buku) edge[arrlabel] node{hyper} (rujukan);
}
\end{scope}

\uncover<5->{
\begin{scope}[font=\footnotesize\color{mLightBrown}]
\node[below of=kamus] {06418901-n};
\node[below of=rujukan] {06417598-n};
\node[below of=buku] {06410904-n};
\end{scope}

}
\end{tikzpicture}
\end{center}
\end{frame}

\begin{frame}
\frametitle{Other WordNet-based Resources}
    
\begin{itemize}[<+->]
\item SentiWordNet \parencite{baccianella2010sentiwordnet}
\begin{itemize}
	\item \url{http://sentiwordnet.isti.cnr.it/}
	\item Provides sentiment scores for each synset
	\item But see also ML-SentiCon \parencite{cruz2014building} \url{http://www.lsi.us.es/~fermin/index.php/Datasets}
\end{itemize}
\item Illustrated WordNet (from Japanese WordNet) \parencite{bond2009enhancing}
	\begin{itemize}
		\item \url{http://wn-msa.sourceforge.net/eng/pics.html}
		\item Provides a clipart for each synset
	\end{itemize}
\item \ldots Many more! Most are OSS.
\end{itemize}

\end{frame}