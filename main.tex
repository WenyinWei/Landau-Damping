\RequirePackage{silence}
\WarningFilter{biblatex}{Patching footnotes failed}
\documentclass[10pt, compress,british,xcolor={svgnames,dvipsnames,x11names},trans]{beamer}

\usepackage{babel}
\usepackage{csquotes}
\usepackage{comment}
\usepackage{tikzsymbols}
% Wenyin used these packages
\usepackage{marginnote}
\usepackage{longtable}
\usepackage{url}
\usepackage{hyperref}
\usepackage{wasysym} % for checkboxes
\usepackage{mathtools} % for overbrace and underbrace
\usepackage{enumerate} % for style setting in itemize
%%% mtheme customisations
\usetheme[progressbar=frametitle,block=fill]{m}
\setmonofont[Scale=0.92]{Fira Mono}
\AtBeginSubsection{
\metroset{color/background=dark}
\frame[plain,c]{
  \begin{center}
  \begin{minipage}{25em}
    \usebeamercolor[fg]{section title}
    \usebeamerfont{section title}
    \insertsubsection\\[-1ex]
    \usebeamertemplate*{progress bar in section page}
  \end{minipage}
  \end{center}
}
\metroset{color/background=light}
}
%%%%% end mtheme

\setbeamertemplate{frametitle continuation}[from second]
\setbeamertemplate{bibliography item}[book]

\usepackage{xeCJK}
\setCJKmainfont{cwTeXFangSong}
% \setCJKsansfont[BoldFont=Kozuka Gothic Pro]{Kozuka Gothic Pro L}
\setCJKsansfont{IPAGothic}
% \newCJKfontfamily{\xiheifont}[BoldFont=STHeiti]{STXihei}
\newCJKfontfamily{\xiheifont}{WenQuanYi Micro Hei}

\usetikzlibrary{arrows}
\usetikzlibrary{chains}
\usepackage{tikz-qtree}
\usepackage{multicol}


\usepackage{expex}
%\lingset{glhangindent=2em,glspace=1em,aboveexskip=0pt,belowexskip=0pt,aboveglftskip=-3pt,extraglskip=3pt} %v0.1
%\lingset{exskip=0pt,interpartskip=-3pt,belowpreambleskip=-3pt,belowglpreambleskip=-3pt,aboveglftskip=-3pt,extraglskip=3pt,glhangstyle=none}
\usepackage{relsize}
\usepackage{booktabs,tabularx}
%\usepackage{textcomp}
\usepackage{listings}
\lstset{basicstyle=\ttfamily,breaklines=true,breakatwhitespace=true,
keywordstyle={\color{NavyBlue}\bfseries}, showstringspaces=false,
commentstyle={\color{PaleVioletRed4}},
emphstyle={\color{OliveGreen}\bfseries}
}

\usepackage{algorithmic}
\renewcommand{\algorithmiccomment}[1]{\alert{/* #1 */}}

\usetikzlibrary{shapes.multipart}
\usetikzlibrary{positioning}
\usetikzlibrary{arrows.meta}

\makeatletter
\pgfarrowsdeclare{crow's foot}{crow's foot}
{
  \pgfarrowsleftextend{+-.5\pgflinewidth}%
  \pgfarrowsrightextend{+.5\pgflinewidth}%
}
{
  \pgfutil@tempdima=0.5pt%
  \advance\pgfutil@tempdima by.25\pgflinewidth%
  \pgfsetdash{}{+0pt}%
  \pgfsetmiterjoin%
  \pgfpathmoveto{\pgfqpoint{0pt}{-6\pgfutil@tempdima}}%
  \pgfpathlineto{\pgfqpoint{-6\pgfutil@tempdima}{0pt}}%
  \pgfpathlineto{\pgfqpoint{0pt}{6\pgfutil@tempdima}}%
  \pgfusepathqstroke%
}

\usepackage[backend=biber,style=apa]{biblatex}
\DeclareLanguageMapping{british}{british-apa}
\renewcommand{\finalnamedelim}{and}
\renewcommand{\bibfont}{\small}
\setlength{\bibhang}{1em}
\setlength{\bibitemsep}{1ex}
\bibliography{refs}
\renewcommand{\UrlFont}{\ttfamily}

\usepackage[os=win]{menukeys}


% \title{Finite Volume Method Implementation in Fluid Dynamics Framework and Corresponding Adaptation Needed for MHD}
% \title{SU2 Code Framework}
\title{Superposition Laser Acceleration PIConGPU toy test}
% \subtitle{With Mature FVM Engine SU2}
\subtitle{Work Done in SNU}

%\date{5 \& 12 March 2015}
%\date{5 March 2015\\Session 1: Common Tasks and Concepts in NLP}
%\date{12 March 2015\\Session 2: Software Libraries and Resources for NLP}
\date{20 Dec 2019}
\author{Wenyin Wei}
\institute{
Engineering Physics Department\\
Tsinghua University Beijing
}

\begin{document}

\maketitle

\begin{frame}{Contents}
\setbeamertemplate{section in toc}[sections numbered]
% \tableofcontents
\tableofcontents[hideallsubsections]
\end{frame}

%!TEX root = main.tex


\section{Basics of Finite Volume Method}

\begin{frame}{Principles of Finite Volume Method}
Firstly, here is a concise introduction to the finite volume method. 
% If you think that is too simple and all audience can understand, just tell me and I would skip.

For the typical equation in the fluid dynamics,
\begin{equation}
\frac{\partial u}{\partial t}+\nabla \cdot \Gamma=F \text { in } \Omega
\end{equation}

Except for the finite difference method which is not compatible with the unstructured mesh, finite volume and finite element are introduced here. These two methods play the role of transforming operators (differentiate, divergence) into discretized parts of an equation. 



\end{frame}

\begin{frame}{Principles of Finite Volume Method}
On the base of the typical equation, a test funciton is multiplied on both sides and the equation is integrated on a specific region, hence,
\begin{equation}
\int_{\Omega} \frac{\partial u}{\partial t} \varphi d V+\int_{\Omega}(\nabla \cdot \Gamma) \varphi d V=\int_{\Omega} F \varphi d V
\end{equation}

By Gauss theorem and integrating by parts, 
\begin{equation}
\int_{\Omega} \frac{\partial u}{\partial t} \varphi d V-\int_{\Omega} \Gamma \cdot \nabla \varphi d V+\int_{\partial \Omega} \Gamma \cdot \mathbf{n} \varphi d S=\int_{\Omega} F \varphi d V
\end{equation}
How to pick the test function and choose the integration region decides whether it is finite volume or finite element method.


\end{frame}

\begin{frame}{Principles of Finite Volume Method}
\begin{equation*}
\int_{\Omega} \frac{\partial u}{\partial t} \varphi d V-\int_{\Omega} \Gamma \cdot \nabla \varphi d V+\int_{\partial \Omega} \Gamma \cdot \mathbf{n} \varphi d S=\int_{\Omega} F \varphi d V
\end{equation*}
In the \alert{finite volume method}, one uses a constant function $1$ directly and integrate the equation on each cell. For each cell, one conservative variable corresponds to an equation. The second term of the function is omitted. The total equation number is nearly $Cell\ Number * Variable\ Number$, the same as FEM.  

While in the \alert{finite element method}, state variables are stored in the nodes of the grid. The integration region is defined as the neighboring region around the node. The test function is chosen to be zero on the boundary of the region so that the third term of the equation is omitted in internel cells.
\end{frame}
\begin{frame}{Principles of Finite Volume Method}
\includegraphics[width=\linewidth]{figures/integrated-flux-internal-cells.png}\\
Finite Volume Method Integration Region Diagram
\end{frame}

\begin{frame}{Principles of Finite Volume Method}
\includegraphics[width=\linewidth]{figures/domain-contribution-internal-elements.png}\\
Finite Element Method Integration Region Diagram
\end{frame}


\begin{frame}[allowframebreaks]
\frametitle{Comparison among CFD and MHD Code}
\footnotesize{\tiny}

\begin{center}
\begin{tabular}{l >{\ttfamily}c >{\ttfamily}c >{\ttfamily}c >{\ttfamily}c}
\toprule
Code & Frame & Institute & Field & Model for MHD \\
\midrule
SU2\footnote{SU2, Stanford University Unstructured \parencite{SU2EconomonIntro}} & FVM/FEM & Stanford U.(open) & CFD &-\\
JOREK & FEM & EUROfusion &   Fusion & 2-fluid \\
M3D-C1\footnote{\textcite{M3DC1:Web}} & FEM & PPPL & Fusion& 2-fluid \\
FLASH &  &  \\
PLUTO & FVM/FDTD & Torino U.(open) & Aero & 1-fluid (RMHD)\\
\bottomrule
\end{tabular}
\end{center}

\begin{center}
\begin{tabular}{l >{\ttfamily}c >{\ttfamily}c >{\ttfamily}c >{\ttfamily}c}
\toprule
Code & Grid & Adaptive Grid & $\cdots$ & $\cdots$ \\
\midrule
SU2 & Unstructured & Support \\
JOREK & Flux-aligned & - & \\
M3D-C1 & Unstructured &\\
FLASH &  &  \\
PLUTO & Structured & Support &\\
\bottomrule
\end{tabular}
\end{center}
\end{frame}

\begin{frame}{Comparison among CFD and MHD Code}
\footnotesize

The most two popular codes in CFD are (1) openFOAM and (2) SU2, in which openFOAM is too large and old for an application in the future. Considering the following points, \alert{SU2} is adopted as the final choice of my research.  

There are some requirements for the research,
\begin{description}
    \item[Modularized]The equations can be coupled flexibly. Especially for multiphysics application, e.g. the plasma-wall interaction is sometimes needed while sometimes omitted for saving time.
    \item[Modern]It would save a lot of time if no fortran embedded.
    \item[Unstructured]Due to the geometry of torus, the grid arrangement is essential for the computation accuracy. Even the flux-aligned 2D Bezier finite element in JOREK may be not precise enough to capture instabilities during the double nut merging due to the flux-aligned grid only has one center. 
    \item[Adaptive Mesh]An adaptive mesh will not only make the result more accurate, but also save tremendous computation resources from those grids which change not so quickly.
\end{description}


\end{frame}


\begin{frame}{JOREK and M3D-C1 Mesh}
\begin{figure}
\begin{minipage}[t]{0.5\linewidth}
\centering
\includegraphics[width=1.2in]{figures/grid_xpoint_medium.png}
\caption{JOREK Mesh}
% \label{fig:side:a}
\end{minipage}%
\begin{minipage}[t]{0.5\linewidth}
\centering
\includegraphics[width=1.8in]{figures/M3D_C1_resistive_wall.png}
\caption{M3D-C1 Mesh}
% \label{fig:side:b}
\end{minipage}
\end{figure}
    
\end{frame}

\begin{frame}{Adaptive Mesh in SU2}
\includegraphics[width=\linewidth]{figures/ONERA_Wing_Initial_Mesh.png}
\end{frame}

\begin{frame}{Adaptive Mesh in SU2}
\includegraphics[width=\linewidth]{figures/ONERA_Wing_Adapted_Mesh.png}
\end{frame}

\subsection{Physics of Codes}

\begin{frame}{Physics of Codes, M3D-C1}
Firstly, a set of fluid equations,
\begin{equation}
\begin{aligned} \frac{\partial n}{\partial t}+\nabla \cdot(n \vec{u})=& 0 \\ n m_{i}\left(\frac{\partial \vec{u}}{\partial t}+\vec{u} \cdot \nabla \vec{u}\right)=& \vec{J} \times \vec{B}-\nabla p-\nabla \cdot \Pi+\vec{F} \\ \frac{\partial p}{\partial t}+\vec{u} \cdot \nabla p+\Gamma p \nabla \cdot \vec{u}=&(\Gamma-1)\left[Q-\nabla \cdot \vec{q}+\eta J^{2}-\vec{u} \cdot \vec{F}-\Pi : \nabla u\right] \\ &+\frac{1}{n e} \vec{J} \cdot\left(\frac{\nabla n}{n} p_{e}-\nabla p_{e}\right)+(\Gamma-1) \Pi_{e} : \nabla\left(\frac{1}{n e} \vec{J}\right) \\ \frac{\partial p_{e}}{\partial t}+\vec{u} \cdot \nabla p_{e}+\Gamma p_{e} \nabla \cdot \vec{u}=&(\Gamma-1)\left[Q_{e}-\vec{q}_{e}+\eta J^{2}-\vec{u} \cdot \vec{F}_{e}-\Pi_{e} : \nabla u\right] \\ &+\frac{1}{n e} \vec{J} \cdot\left(\frac{\nabla n}{n} p_{e}-\nabla p_{e}\right)+(\Gamma-1)\left[\Pi_{e} : \nabla\left(\frac{1}{n e} \vec{J}\right)+\frac{1}{n e} \vec{J} \cdot \vec{F}_{e}\right] \end{aligned}
\end{equation}

Generalized Ohm's law,
\begin{equation}
\vec{E}=-\vec{u} \times \vec{B}+\eta \vec{J}+\frac{1}{n e}\left(\vec{J} \times \vec{B}-\nabla p_{e}-\nabla \cdot \Pi_{e}+\vec{F}_{e}\right)
\end{equation}

A reduced set of Maxwell's equations,
\begin{equation}
\begin{aligned} \vec{J} &=\frac{1}{\mu_{0}} \nabla \times \vec{B} \\ \frac{\partial \vec{B}}{\partial t} &=-\nabla \times \vec{E} \end{aligned}
\end{equation}
\end{frame}

\begin{frame}{Physics of Codes, JOREK}

2-fluid MHD model is probably a too large topic to talk about in this presentation, because it is not the focus of the presentation.

Here we have the resitivity MHD model in JOREK as an example.

\alert{Resitivity MHD model},
\begin{equation}
\left\{
\begin{array}{l}
{\partial_{t} \rho+\nabla \cdot(\rho \boldsymbol{v})=0} \\ 
{\rho \partial_{t} \boldsymbol{v}+\rho \boldsymbol{v} \cdot \nabla \boldsymbol{v}+\nabla(p)=\boldsymbol{J} \times \boldsymbol{B}+\nabla \cdot(\nu \nabla \boldsymbol{v})} \\ 
{\partial_{t} p+\boldsymbol{v} \cdot \nabla p+\gamma p \nabla \cdot \boldsymbol{v}=0} \\ {\partial_{t} \boldsymbol{B}=-\nabla \times \boldsymbol{E}=\nabla \times(\boldsymbol{v} \times \boldsymbol{B}-\eta \boldsymbol{J})} \\ {\nabla \times \boldsymbol{B}=\boldsymbol{J}} \\ {\nabla \cdot \boldsymbol{B}=0}\end{array}\right.
\end{equation}

Be careful of the degenerative Maxwell:
\begin{itemize}
    \item $\boldsymbol{J}$ is decided by the curl of magnetic field here, which means $\boldsymbol{E}$ is much more stable than $\boldsymbol{B}$.
    \item Time derivative of $p$ is replaced by the energy derivative in fluid dynamics SU2. 
\end{itemize}


\end{frame}


\begin{frame}{Physics of Codes, SU2}


\begin{equation*}
\frac{\partial U}{\partial t}+\nabla \cdot \vec{F}^{c}-\nabla \cdot\left(\mu^{v k} \vec{F}^{v k}\right)=Q \quad \text { in } \Omega, t>0
\end{equation*}


\begin{equation*}
U=\left\{\begin{array}{c}{\rho } \\ {\rho \vec{v} } \\ {\rho E}\end{array}\right\},\quad
\vec{F}^{c}=\left\{\begin{array}{c}{\rho \vec{v}} \\ {\rho \vec{v} \otimes \vec{v}+\overline{\overline{I}} p} \\ {\rho E \vec{v}+p \vec{v}}\end{array}\right\}
\end{equation*}

\begin{equation*}
\vec{F}^{v1}=\left\{\begin{array}{c}{\cdot} \\ {\overline{\overline{\tau}}} \\ {\overline{\overline{\tau}}\cdot \vec{v}}\end{array}\right\},\quad
\vec{F}^{v2}=\left\{\begin{array}{c}{\cdot} \\ {\cdot} \\ {c_p \nabla T}\end{array}\right\},\quad
Q=\left\{\begin{array}{c}{q_\rho} \\ {\vec{q}_{\rho\vec{v}}} \\ {q_{\rho E}}\end{array}\right\}
\end{equation*}

\end{frame}

\begin{frame}{Conservative Equations}
\begin{equation}
\frac{\partial \boldsymbol{U}}{\partial t}+\underbrace{\nabla \cdot \boldsymbol{F}^{c}}_{\text CONV\ TERM}-\overbrace{\nabla \cdot\left(\mu^{v k} \boldsymbol{F}^{v k}\right)}^{\text VISC\ TERM}=\boldsymbol{Q} \quad \text { in }\Omega,  \quad t>0
\end{equation}

Most SU2 scripts are located in the \textit{SU2\_CFD} and \textit{Common} folder, other folders contain very limited amount of codes concerning adaptive mesh, dual grid and \textit{e.t.c.} 

\begin{itemize}
    \item OS: Cross-platform
    \item Lang: C++, Python
    \item Mainly vertex-based dual mesh FVM, also supports FEM and primal grid
\end{itemize}
\end{frame}

\begin{frame}{SU2 Framework}
\uncover<+->{\begin{equation*}
\frac{\partial U}{\partial t}+\nabla \cdot \vec{F}^{c}-\nabla \cdot\left(\mu^{v k} \vec{F}^{v k}\right)=Q \quad \text { in } \Omega, t>0
\end{equation*}
}
\uncover<+->{\[ \Downarrow \]
\begin{align*}
    \int_{\Omega_{i}} \frac{\partial U}{\partial t} d \Omega & + \sum_{j \in \mathcal{N}(i)}\left(\tilde{F}_{i j}^{c}+\tilde{F}_{i j}^{v k}\right) \Delta S_{i j}-Q\left|\Omega_{i}\right| \\ 
=\int_{\Omega_{i}} \frac{\partial U}{\partial t} d \Omega & +  R_{i}(U)=0
\end{align*}
}

For each variable in each cell, there is a residual.
\end{frame}



% \subsection{Temporal Discretization}
\begin{frame}{Temporal Discretization}
The requirement of temporal discretization in Maxwell equations is no different from others.

Explicit RK,
\begin{equation}
\left\{\begin{array}{l}{u^{(1)}=u^{n}} \\ {u^{(2)}=u^{n}+\alpha_{2} d t F\left(u^{(1)}\right)} \\ {u^{(3)}=u^{n}+\alpha_{3} d t F\left(u^{(2)}\right)} \\ {\cdots} \\ {u^{(k)}=u^{n}+\alpha_{k} d t F\left(u^{(k-1)}\right)} \\ {u^{n+1}=u^{n}+d t \sum_{j=1}^{q} \beta_{j} F\left(u^{(j)}\right)}\end{array}\right.
\end{equation}

Implicit RK and dual time stepping \parencite{SU2LiDualTime} are also available in SU2, and compatible with the Maxwell equations.
\end{frame}





% \begin{frame}[fragile,allowframebreaks]
% \frametitle{Java Code Sample}
    
% \begin{lstlisting}[language=Java,basicstyle=\ttfamily\footnotesize,gobble=8,
%     emph={parse,lp,typedDependenciesCCprocessed,lemmaStatic,dep,gov,reln,index,tag,value},
%     morekeywords={TreebankLanguagePack,GrammaticalStructureFactory,GrammaticalStructure,
%         List,TypedDependency,IndexedWord,String,Morphology},
%         escapechar=|]
%         // Continue from earlier Java code 
%         // Use the parsed tree to get the typed dependencies
%         TreebankLanguagePack tlp = lp.treebankLanguagePack();
%         GrammaticalStructureFactory gsf = tlp.grammaticalStructureFactory();
%         GrammaticalStructure gs = gsf.newGrammaticalStructure(parse);
%         List<TypedDependency> tdl = gs.typedDependenciesCCprocessed();

%         // Let's just print out each of the parent-child relationship first
%         for (TypedDependency td : tdl) {
%             // parent = "governer"
%             IndexedWord parent = td.gov();
%             String parentWord = parent.value();
%             String parentPOS = parent.tag();
%             String parentLemma = Morphology.lemmaStatic(
%                 parentWord, parentPOS, true);
            
%             |\framebreak|
%             // child = "dependent"
%             IndexedWord child = td.dep();
%             String childWord = child.value();
%             String childPOS = child.tag();
%             String childLemma = Morphology.lemmaStatic(
%                 childWord, childPOS, true);
            
            
%             System.out.println(
%                 "[" + parent.index() + "]" + parentLemma + "/" + parentPOS 
%                 + " <--" + td.reln().getShortName() + "-- "
%                 + "[" + child.index() + "]" + childLemma + "/" + childPOS);
%         }
%         System.out.println();
% \end{lstlisting}

% \end{frame}








\begin{frame}{Vertex-based Dual Grid}
\begin{figure}
    \centering
    \includegraphics[width=0.5\linewidth]{figures/DualMeshControlVolume.png}
    \caption{Dual mesh control volumes surrounding two nodes, $i$ and $j$, in the domain interior.}
    % \label{fig:my_label}
\end{figure}


By virtue of the functionality of geometry-relevant source, researchers do not need to implement the area calculation step and are capable to use the face area and normal vector directly.

\end{frame}

%!TEX root = Intro-NLP-seminar.tex

\section{Maxwell Equations in FVM}

% \subsection{Denotations and Equations used in Maxwell-FVM}
\begin{frame}{Maxwell Equations}
Maxwell equations indicate that the $\boldsymbol{E}$ and $\boldsymbol{H}$ changes are decided by the curl of the other variable. \parencite{rao1999time} That is unprecedented in other fluid equations, which brings new numerical trouble.

\begin{equation}
\left\{\begin{array}{l}{\nabla \times \boldsymbol{H}-\varepsilon \frac{\partial \boldsymbol{E}}{\partial t}-\sigma \boldsymbol{E}=\boldsymbol{J}} \\ {\nabla \times \boldsymbol{E}+\mu \frac{\partial \boldsymbol{H}}{\partial t}=\boldsymbol{K}}\end{array}\right.
\end{equation}

The following continuity conditions always work for any interface,

\begin{equation}
\begin{array}{r}{\left[\boldsymbol{E}_{\mathrm{\tau}}, \boldsymbol{H}_{\mathrm{\tau}}\right]_{-}^{+}=0} \\ {\left[\varepsilon \boldsymbol{a}_{\mathrm{n}} \cdot \boldsymbol{E}, \mu \boldsymbol{a}_{\mathrm{n}} \cdot \boldsymbol{H}\right]_{-}^{+}=0}\end{array}
\end{equation}

The $[\ \cdot\ ]^+_-$ symbols mean the difference between the positive side of the interface and the negative.

\end{frame}

% \begin{frame}{Maxwell Equations}

% In the special cases of impenetrable obstacles, one would take $\boldsymbol{e} = 0$ on perfect conductors and $\boldsymbol{h} = 0$ on perfectly magnetizable media, respectively.

% \begin{equation}
% \left\{\begin{array}{l}{\left.\boldsymbol{E}_{\mathrm{\tau}}\right|_{\partial \Omega}=\boldsymbol{e}} \\ {\left.\boldsymbol{H}_{\mathrm{\tau}}\right|_{\partial \Omega}=\boldsymbol{h}}\end{array}\right.
% \end{equation}

% The $\boldsymbol{e} = 0$ boundary condition is a common one in electromagnetic simulation, but primitive. More advanced boundary to simulate far-field condition will be illustrated subsequently.

% \end{frame}

% \begin{frame}{Maxwell Equations}

% For convenience to do computation, $s$ is denoted as a function mapping a vector to an antisymmetric 3 x 3 matrices.
% \begin{equation}
% s : R^{3} \rightarrow \Lambda^{2}\left(R^{3}\right)
% \end{equation}


% \begin{equation}
% s :\left(v_{1}, v_{2}, v_{3}\right) \longmapsto\left[\begin{array}{ccc}{0} & {-v_{3}} & {v_{2}} \\ {v_{3}} & {0} & {-v_{1}} \\ {-v_{2}} & {v_{1}} & {0}\end{array}\right]
% \end{equation}

% Special divergence is also defined, for handling the divergence of a 'matrix'.
% \begin{equation}
% [\operatorname{Div}(s(\boldsymbol{V}))]_{q}=\sum_{p=1}^{3} \frac{\partial}{\partial x^{p}}[s(\boldsymbol{V})]_{p q}
% \end{equation}


% \end{frame}

% \begin{frame}{Maxwell Equations}
% There are two essential characteristics of this funciton.

% Firstly, to calculate the curl of a vector-value variable by a divergence operator is not availiable. 
% \begin{equation}
% \operatorname{Div}(s(\boldsymbol{V}))=\nabla \times \boldsymbol{V}
% \end{equation}


% Secondly, it makes it easier to do cross-product.
% \begin{equation}
% s(\boldsymbol{a}_n)\boldsymbol{V}= \boldsymbol{a}_n \times \boldsymbol{V}
% \end{equation}


% \end{frame}



% \begin{frame}{Maxwell Equations}
% \begin{equation}
% \left\{\begin{array}{l}{\varepsilon \frac{\partial \boldsymbol{E}}{\partial t}-\operatorname{Div}(s(\boldsymbol{H}))+\sigma \boldsymbol{E}=-\boldsymbol{J}} \\ {\mu \frac{\partial \boldsymbol{H}}{\partial t}+\operatorname{Div}(s(\boldsymbol{E}))=\boldsymbol{K}}\end{array}\right.
% \end{equation}


% \begin{equation}
% \alpha \frac{\partial \boldsymbol{U}}{\partial t}+\operatorname{div}(\mathcal{A} \boldsymbol{U})+\mathcal{B} \boldsymbol{U}=\boldsymbol{G}
% \end{equation}

% Attention, the $\operatorname{div}$ operates on each row of 6 x 3 matrix and acquire a 6 x 1 vector, different from the $\operatorname{Div}$ operator.

% \begin{equation}
% \begin{array}{c}{[\alpha]=\left[\begin{array}{cc}{\varepsilon} & {0} \\ {0} & {\mu}\end{array}\right] \quad[\mathcal{A}]=\left[\begin{array}{cc}{0} & {-s(\cdot)} \\ {s(\cdot)} & {0}\end{array}\right] \quad[\mathcal{B}]=\left[\begin{array}{cc}{\sigma} & {0} \\ {0} & {0}\end{array}\right]} \\ {\boldsymbol{G}=\left[\begin{array}{c}{-\boldsymbol{J}} \\ {\boldsymbol{K}}\end{array}\right]}\end{array}
% \end{equation}

% \begin{equation}
% \mathcal{A} U=F(U)=\left(F_{1}(U) ; F_{2}(U) ; F_{3}(U)\right)
% \end{equation}

% \end{frame}

% \begin{frame}{Maxwell Equation, Spatial Discretization}
% Let's put the current density aside, which is not a big problem after the integration.

% Then it becomes the \alert{Conservative Form Maxwell equations}:

% \begin{equation}
% \begin{aligned} \varepsilon \frac{\partial \boldsymbol{E}}{\partial t}-\nabla \times \boldsymbol{H} &=0 \\ \mu \frac{\partial \boldsymbol{H}}{\partial t}+\nabla \times \boldsymbol{E} &=0 \end{aligned}
% \end{equation}



% \begin{equation}
% \mathcal{A} \boldsymbol{U}=F(\boldsymbol{U})=\left(F_{1}(\boldsymbol{U}) ; F_{2}(\boldsymbol{U}) ; F_{3}(\boldsymbol{U})\right)
% \end{equation}

% \end{frame}



% \begin{frame}{Maxwell Equations}
% \begin{equation}
% F_{1}(\boldsymbol{U})=\left[\begin{array}{c}{0} \\ {H_{2}} \\ {-H_{y}} \\ {0} \\ {-E_{z}} \\ {E_{y}}\end{array}\right] \quad F_{2}(\boldsymbol{U})=\left[\begin{array}{c}{-H_{z}} \\ {0} \\ {H_{x}} \\ {E_{z}} \\ {0} \\ {-E_{x}}\end{array}\right] \quad F_{3}(\boldsymbol{U})=\left[\begin{array}{c}{H_{y}} \\ {-H_{x}} \\ {0} \\ {-E_{y}} \\ {E_{x}} \\ {0}\end{array}\right]
% \end{equation}

% \begin{equation}
% \alpha \frac{\partial \boldsymbol{U}}{\partial t}+\frac{\partial F_{1}(\boldsymbol{U})}{\partial x}+\frac{\partial F_{2}(\boldsymbol{U})}{\partial y}+\frac{\partial F_{3}(\boldsymbol{U})}{\partial z}=0
% \end{equation}

% \begin{equation}
% \alpha \frac{\partial \boldsymbol{U}}{\partial t}+\operatorname{div} F(\boldsymbol{U})=0
% \end{equation}

% \end{frame}

\begin{frame}{Maxwell Equation, Spatial Discretization}

\begin{equation}
\alpha \int_{V} \frac{\partial \boldsymbol{U}}{\partial t} d V+\int_{V} \operatorname{div} F(\boldsymbol{U}) d \boldsymbol{V}=0
\end{equation}

\begin{equation}
\frac{\partial \boldsymbol{U}_{i}}{\partial t}=-\frac{1}{\left|V_{i}\right|} \sum_{k\in\mathcal{N}_i}\left|S_{k}\right| \alpha^{-1} F\left(\boldsymbol{U}_{k}^{*}\right) \cdot \boldsymbol{a}_{\mathrm{nk}}
\end{equation}

\begin{equation}
F\left(\boldsymbol{U}_{k}^{*}\right) \cdot \boldsymbol{a}_{n k}=\left[\begin{array}{c}{-\boldsymbol{a}_{n k} \times \boldsymbol{H}_{k}^{*}} \\ {\boldsymbol{a}_{n k} \times \boldsymbol{E}_{k}^{*}}\end{array}\right]
\end{equation}
    
The spatial discretization methods focus on the key variable, $F\left(\boldsymbol{U}_{k}^{*}\right) \cdot \boldsymbol{a}_{\mathrm{nk}}$. How to accurately compute the value of the flux is the kernel of FVM, that is, computing the $\boldsymbol{a}_{n k} \times \boldsymbol{H}_{k}^{*}, \boldsymbol{a}_{n k} \times \boldsymbol{E}_{k}^{*}$.
\end{frame}

% \subsection{How to Compute the Flux in Maxwell Equations}
\begin{frame}{$F\left(\boldsymbol{U}_{k}^{*}\right) \cdot \boldsymbol{a}_{\mathrm{nk}}$}
How to calculate the flux, $F\left(\boldsymbol{U}_{k}^{*}\right) \cdot \boldsymbol{a}_{\mathrm{nk}}$? The values on the cell boundary are needed, but we only store the values on the nodes.

Then one might come up with these ideas,

\begin{description}
\item[\textbf{Interpolation}] By utility of the value interpolation from the values at adjacent nodes, the flux is easy to get and have a high-order accuracy. It would be quite useful for the continuously differentiable functions but probably not for $\varepsilon$ or $\mu$ on the boundary of materials. 
\item[\textbf{Flux-Splitting}] To settle the problem arising from numerical instability of other methods, flux-splitting method is used.
\end{description}
The following content related to the flux-splitting method begins with a matrix constructed by the $\boldsymbol{a}_{\mathrm{n}}$.

\end{frame}

\begin{frame}{$F\left(\boldsymbol{U}_{k}^{*}\right) \cdot \boldsymbol{a}_{\mathrm{nk}}$, $A$ Matrix}
One can use the normal vectors of a cell's faces to represent the local coordinates, as a basis of coordinate,
\begin{equation}
R^{m_{i}} \ni\left(\xi_{1}, \cdots, \xi_{m_{i}}\right) \mapsto x=\sum_{k=1}^{m_{i}} \xi_{k} a_{\mathrm{nk}} \in R^{3}
\end{equation}

\begin{equation}
\frac{\partial \boldsymbol{U}}{\partial t}=-\alpha^{-1} \sum_{k=1}^{m_{i}}\left(\frac{\partial F_{1}}{\partial \boldsymbol{U}} \frac{\partial \xi_{k}}{\partial x}+\frac{\partial F_{2}}{\partial \boldsymbol{U}} \frac{\partial \xi_{k}}{\partial y}+\frac{\partial F_{3}}{\partial \boldsymbol{U}} \frac{\partial \xi_{k}}{\partial z}\right) \frac{\partial \boldsymbol{U}}{\partial \xi_{k}}
\end{equation}
\begin{equation}
\frac{\partial \boldsymbol{U}}{\partial t}=-\alpha^{-1} \sum_{k=1}^{m_{i}} A\left(\boldsymbol{a}_{\mathrm{nk}}\right) \frac{\partial \boldsymbol{U}}{\partial n_{k}}
\end{equation}
\end{frame}


\begin{frame}{$F\left(\boldsymbol{U}_{k}^{*}\right) \cdot \boldsymbol{a}_{\mathrm{nk}}$, $A$ Matrix}
\begin{equation}
A\left(\boldsymbol{a}_{\mathrm{n}}\right)=\left[\begin{array}{cccccc}{0} & {0} & {0} & {0} & {n_{z}} & {-n_{y}} \\ {0} & {0} & {0} & {-n_{z}} & {0} & {n_{x}} \\ {0} & {0} & {0} & {n_{y}} & {-n_{x}} & {0} \\ {0} & {-n_{z}} & {n_{y}} & {0} & {0} & {0} \\ {n_{z}} & {0} & {-n_{x}} & {0} & {0} & {0} \\ {-n_{y}} & {n_{x}} & {0} & {0} & {0} & {0}\end{array}\right]
\end{equation}

\begin{equation}
\tilde{A}\left(\boldsymbol{a}_{\mathrm{n}}\right)=\alpha^{-1} A\left(\boldsymbol{a}_{\mathrm{n}}\right)=P\ \Lambda\ P^{-1},
\end{equation}

with six eigenvalues also as the diagonal of $\Lambda$
\begin{equation}
\Lambda = diag \left(0,0, \frac{1}{\sqrt{\mu \varepsilon}}, \frac{1}{\sqrt{\mu \varepsilon}},-\frac{1}{\sqrt{\mu \varepsilon}},-\frac{1}{\sqrt{\mu \varepsilon}}\right) = diag \left(0,0,v,v,-v,-v\right),
\end{equation}
where $v$ is the light speed in the medium. 
\end{frame}

\begin{frame}{Maxwell Equations}

For convenience to do computation, $s$ is denoted as a function mapping a vector to an antisymmetric 3 x 3 matrices.
\begin{equation}
s : R^{3} \rightarrow \Lambda^{2}\left(R^{3}\right)
\end{equation}


\begin{equation}
s :\left(v_{1}, v_{2}, v_{3}\right) \longmapsto\left[\begin{array}{ccc}{0} & {-v_{3}} & {v_{2}} \\ {v_{3}} & {0} & {-v_{1}} \\ {-v_{2}} & {v_{1}} & {0}\end{array}\right]
\end{equation}

Special divergence is also defined, for handling the divergence of a 'matrix'.
\begin{equation}
[\operatorname{Div}(s(\boldsymbol{V}))]_{q}=\sum_{p=1}^{3} \frac{\partial}{\partial x^{p}}[s(\boldsymbol{V})]_{p q}
\end{equation}


\end{frame}

\begin{frame}{Maxwell Equations}
There are two essential characteristics of this funciton.

Firstly, to calculate the curl of a vector-value variable by a divergence operator is not availiable. 
\begin{equation}
\operatorname{Div}(s(\boldsymbol{V}))=\nabla \times \boldsymbol{V}
\end{equation}


Secondly, it makes it easier to do cross-product.
\begin{equation}
s(\boldsymbol{a}_n)\boldsymbol{V}= \boldsymbol{a}_n \times \boldsymbol{V}
\end{equation}


\end{frame}
\begin{frame}{$F\left(\boldsymbol{U}_{k}^{*}\right) \cdot \boldsymbol{a}_{\mathrm{nk}}$, $A$ Matrix}
\begin{equation}
P=\left[\begin{array}{cccccc}{n_{x}} & {0} & {\frac{n_{x} n_{z}}{v \varepsilon}} & {-\frac{n_{x} n_{y}}{v \varepsilon}} & {-\frac{n_{x} n_{z}}{v \varepsilon}} & {\frac{n_{x} n_{y}}{v \varepsilon}} \\ {n_{y}} & {0} & {\frac{n_{y} n_{z}}{v \varepsilon}} & {\frac{n_{x}^{2}+n_{z}^{2}}{v \varepsilon}} & {-\frac{n_{y} n_{z}}{v \varepsilon}} & {-\frac{n_{x}^{2}+n_{z}^{2}}{v \varepsilon}} \\ {n_{z}} & {0} & {-\frac{n_{x}^{2}+n_{y}^{2}}{v \varepsilon}} & {-\frac{n_{y} n_{z}}{v \varepsilon}} & {\frac{n_{x}^{2}+n_{y}^{2}}{v \varepsilon}} & {\frac{n_{y} n_{z}}{v \varepsilon}} \\ {0} & {n_{x}} & {-n_{y}} & {-n_{z}} & {-n_{y}} & {-n_{z}} \\ {0} & {n_{y}} & {n_{x}} & {0} & {n_{x}} & {0} \\ {0} & {n_{z}} & {0} & {n_{x}} & {0} & {n_{x}}\end{array}\right]
\end{equation}

If we consider the variation of U to be only with respect to the $\tilde{A}\left(\boldsymbol{a}_{\mathrm{n}}\right)$ direction, then equation simplies to,
\begin{equation}
\frac{\partial \boldsymbol{U}}{\partial t}=-\tilde{A}\left(\boldsymbol{a}_{\mathrm{n}}\right) \frac{\partial \boldsymbol{U}}{\partial n}
\end{equation}

Using the previous decomposition of $\tilde{A}\left(\boldsymbol{a}_{\mathrm{n}}\right)$ and introducing the vector $V = P^{-1}U$
\begin{equation}
\frac{\partial V_{j}}{\partial t}+\lambda_{j} \frac{\partial V_{j}}{\partial n}=0 \quad j=1, \cdots, 6
\end{equation}

\end{frame}

\begin{frame}{Maxwell Equations, Flux-Splitting}
\begin{equation}
V_{j}(\xi, t)=f\left(\lambda_{j} t-\xi n\right),
\end{equation}
\begin{equation}
V_{j}\left(\xi_{0}, t-\frac{d}{\lambda_{j}}\right)=V_{j}\left(\xi^{*}, t\right)
\end{equation}

A crude approximation would be like the following.
\begin{equation}\label{VApproxBoundary}
V_{j}(t)=V_{j}^{*}(t)
\end{equation}

\begin{equation}
\tilde{A}\left(\boldsymbol{a}_{\mathrm{n}}\right)=P \Lambda P^{-1}=P\left(\Lambda^{+}+\Lambda^{-}\right) P^{-1}=P \Lambda^{+} P^{-1}+P \Lambda^{-} P^{-1}=\tilde{A}\left(\boldsymbol{a}_{\mathrm{n}}\right)^{+}+\tilde{A}\left(\boldsymbol{a}_{\mathrm{n}}\right)^{-}
\end{equation}

Multiply both sides of equation (\ref{VApproxBoundary}) by $P\Lambda^+$
\begin{equation}
\tilde{A}\left(\boldsymbol{a}_{\mathrm{n}}\right)^{+} \boldsymbol{U}=\tilde{A}\left(\boldsymbol{a}_{\mathrm{n}}\right)^{+} \boldsymbol{U}^{*}
\end{equation}
\end{frame}

\begin{frame}{Maxwell Equations, Flux-Splitting}
\tiny
\begin{equation}
\tilde{A}\left(\boldsymbol{a}_{\mathrm{n}}\right)^{+}=\frac{1}{2}\left[\begin{array}{cccccc}{\left(n_{z}^{2}+n_{y}^{2}\right) v} & {-n_{x} n_{y} v} & {-n_{x} n_{z} v} & {0} & {n_{z} \frac{1}{\varepsilon}} & {-n_{y} \frac{1}{\varepsilon}} \\ {-n_{x} n_{y} v} & {\left(n_{x}^{2}+n_{z}^{2}\right) v} & {-n_{y} n_{z} v} & {-n_{z} \frac{1}{\varepsilon}} & {0} & {n_{x} \frac{1}{\varepsilon}} \\ {-n_{x} n_{z} v} & {-n_{y} n_{z} v} & {\left(n_{x}^{2}+n_{y}^{2}\right) v} & {n_{y} \frac{1}{\varepsilon}} & {-n_{x} \frac{1}{\varepsilon}} & {0} \\ {0} & {-n_{z} \frac{1}{\mu}} & {n_{y} \frac{1}{\mu}} & {\left(n_{z}^{2}+n_{y}^{2}\right) v} & {-n_{x} n_{y} v} & {-n_{x} n_{z} v} \\ {n_{z} \frac{1}{\mu}} & {0} & {-n_{x} \frac{1}{\mu}} & {-n_{x} n_{y} v} & {\left(n_{x}^{2}+n_{z}^{2}\right) v} & {-n_{y} n_{z} v} \\ {-n_{y} \frac{1}{\mu}} & {n_{x} \frac{1}{\mu}} & {0} & {-n_{x} n_{z} v} & {-n_{y} n_{z} v} & {\left(n_{x}^{2}+n_{y}^{2}\right) v}\end{array}\right]
\end{equation}
\normalsize

\begin{equation}
\tilde{A}\left(\boldsymbol{a}_{\mathrm{n}}\right)^{-}=-\tilde{A}\left(-\boldsymbol{a}_{\mathrm{n}}\right)^{+}
\end{equation}

\begin{equation}
\tilde{A}\left(\boldsymbol{a}_{\mathrm{n}}\right)^{+}=\frac{1}{2}\begin{bmatrix}
-vs^2(\boldsymbol{a}_n) & \frac{-1}{\varepsilon} s(\boldsymbol{a}_n) \\ 
\frac{1}{\mu} s(\boldsymbol{a}_n) & -vs^2(\boldsymbol{a}_n
\end{bmatrix},\quad \tilde{A}\left(\boldsymbol{a}_{\mathrm{n}}\right)^{+} \boldsymbol{U}=\tilde{A}\left(\boldsymbol{a}_{\mathrm{n}}\right)^{+} \boldsymbol{U}^{*}
\end{equation}




\begin{equation}
Y=\sqrt{\frac{\varepsilon}{\mu}}, Z=\sqrt{\frac{\mu}{\varepsilon}}
\end{equation}
\end{frame}


\begin{frame}{Maxwell Equations, Flux-Splitting}
For \textbf{dielectric contrast},
\begin{equation}
\left\{\begin{array}{l}{\boldsymbol{a}_{\mathbf{n}} \times \boldsymbol{E}^{*}=\boldsymbol{a}_{\mathbf{n}} \times \boldsymbol{E}^{* *}} \\ {\boldsymbol{a}_{\mathbf{n}} \times \boldsymbol{H}^{*}=\boldsymbol{a}_{\mathbf{n}} \times \boldsymbol{H}^{* *}}\end{array}\right.
\end{equation}


\begin{equation}
\left\{\begin{array}{l}{Y^{\mathrm{L}} \boldsymbol{a}_{\mathrm{n}} \times \boldsymbol{E}^{*}-\boldsymbol{a}_{\mathrm{n}} \times\left(\boldsymbol{a}_{\mathrm{n}} \times \boldsymbol{H}^{*}\right)=Y^{\mathrm{L}} \boldsymbol{a}_{\mathrm{n}} \times \boldsymbol{E}^{\mathrm{L}}-\boldsymbol{a}_{\mathrm{n}} \times\left(\boldsymbol{a}_{\mathrm{n}} \times \boldsymbol{H}^{\mathrm{L}}\right)} \\ {Y^{\mathrm{R}} \boldsymbol{a}_{\mathrm{n}} \times \boldsymbol{E}^{* *}+\boldsymbol{a}_{\mathrm{n}} \times\left(\boldsymbol{a}_{\mathrm{n}} \times \boldsymbol{H}^{* *}\right)=Y^{\mathrm{R}} \boldsymbol{a}_{\mathrm{n}} \times \boldsymbol{E}^{\mathrm{R}}+\boldsymbol{a}_{\mathrm{n}} \times\left(\boldsymbol{a}_{\mathrm{n}} \times \boldsymbol{H}^{\mathrm{R}}\right)}\end{array}\right.
\end{equation}


\begin{equation}
F\left(\boldsymbol{U}^{*}\right) \cdot \boldsymbol{a}_{\mathrm{n}}=F\left(\boldsymbol{U}^{* *}\right) \cdot \boldsymbol{a}_{\mathrm{n}}=\alpha^{\mathrm{L}} T^{\mathrm{L}} \tilde{A}^{\mathrm{L}}\left(\boldsymbol{a}_{\mathrm{n}}\right)^{+} U^{\mathrm{L}}+\alpha^{\mathrm{R}} T^{\mathrm{R}} \tilde{A}^{\mathrm{R}}\left(\boldsymbol{a}_{\mathrm{n}}\right)^{-} \boldsymbol{U}^{\mathrm{R}}
\end{equation}

\begin{equation}
T^{\mathrm{L}, \mathrm{R}}=\left[\begin{array}{cc}{\frac{2 Z^{\mathrm{L} \cdot R}}{Z^{\mathrm{L}}+Z^{\mathrm{R}}}} & {0} \\ {0} & {\frac{2 Y^{L, R}}{Y^{\mathrm{L}}+Y^{\mathrm{R}}}}\end{array}\right]
\end{equation}





\end{frame}

\begin{frame}{Maxwell Equations, Flux-Splitting}
\begin{equation}
\alpha \frac{\partial U_{i}}{\partial t}=\frac{-1}{\left|V_{i}\right|} \sum_{k\in\mathcal{N}_i}\left|S_{k}\right|\left[\alpha^{\mathrm{L}} T^{\mathrm{L}} \tilde{A}^{\mathrm{L}}\left(\boldsymbol{a}_{\mathrm{n} k}\right)^{+} U_{k^{-}}+\alpha^{\mathrm{R}} T^{\mathrm{R}} \tilde{A}^{\mathrm{R}}\left(\boldsymbol{a}_{\mathrm{nk}}\right)^{-} U_{k^{+}}\right]
\end{equation}

If both sides' permittivity and peameability are similar such that the error could be omitted, the formula turns out to be,
\begin{equation}
\frac{\partial U_{i}}{\partial t}=\frac{-1}{\left|V_{i}\right|} \sum_{k\in\mathcal{N}_i}\left|S_{k}\right|\left[\tilde{A}\left(\boldsymbol{a}_{\mathrm{nk}}\right)^{+} U_{k^{-}}+\tilde{A}\left(\boldsymbol{a}_{\mathrm{nk}}\right)^{-} U_{k^{+}}\right]
\end{equation}
\end{frame}

\begin{frame}{Maxwell Equations, Flux-Splitting}
For \textbf{perfect conductor},
\begin{equation}
\boldsymbol{a}_{\mathrm{n}} \times \boldsymbol{E}^{*}=0
\end{equation}

\begin{equation}
\boldsymbol{a}_{\mathrm{n}} \times \boldsymbol{H}^{*}=Y \boldsymbol{a}_{\mathrm{n}} \times\left(\boldsymbol{a}_{\mathrm{n}} \times \boldsymbol{E}\right)+\boldsymbol{a}_{\mathrm{n}} \times \boldsymbol{H}
\end{equation}

\begin{equation}
F\left(\boldsymbol{U}^{*}\right) \cdot \boldsymbol{a}_{\mathrm{n}}=\alpha^{\mathrm{L}} T_{p c}^{\mathrm{L}} \tilde{A}^{\mathrm{L}}\left(\boldsymbol{a}_{\mathrm{n}}\right)^{+} \boldsymbol{U}^{\mathrm{L}}
\end{equation}

\begin{equation}
T_{p c}^{\mathrm{L}}=\lim _{Z^{\mathrm{R}} \rightarrow 0} T^{\mathrm{L}}
\end{equation}

\begin{equation}
\left[T_{p c}^{\mathrm{L}}\right]=\left[\begin{array}{cc}{2 \mathrm{Id}} & {0} \\ {0} & {0}\end{array}\right]
\end{equation}

\end{frame}

\begin{frame}{Maxwell Equations, Flux-Splitting}
\textbf{Radiation Boundary Condition}
\begin{description}
    \item[Silver-Muller] The condition is simply that the flux on the outer boundary corresponds
to outgoing waves only. By the following equation, it is enough to calculate the result.

\begin{equation}
\sqrt{\frac{\varepsilon_{0}}{\mu_{0}}} \boldsymbol{a}_{\mathrm{n}} \times \boldsymbol{E}^{*}+\boldsymbol{a}_{\mathrm{n}} \times\left(\boldsymbol{a}_{\mathrm{n}} \times \boldsymbol{H}^{*}\right)=0
\end{equation}
    \item[PML] Perfectly Matched Layer. Compared to Silver-Muller condition, PML is more like an active operation to decrease the noise reflecting electromagnetic field.
    \begin{equation}
\frac{\sigma}{\varepsilon_{0}}=\frac{\sigma^{*}}{\mu_{0}}
\end{equation}
\end{description}

\end{frame}


% \section{Structure of SU2}

\begin{frame}{Conservative Equations}
\begin{equation}
\frac{\partial \boldsymbol{U}}{\partial t}+\underbrace{\nabla \cdot \boldsymbol{F}^{c}}_{\text CONV\ TERM}-\overbrace{\nabla \cdot\left(\mu^{v k} \boldsymbol{F}^{v k}\right)}^{\text VISC\ TERM}=\boldsymbol{Q} \quad \text { in }\Omega,  \quad t>0
\end{equation}

Most SU2 scripts are located in the \textit{SU2\_CFD} and \textit{Common} folder, other folders contain very limited amount of codes concerning adaptive mesh, dual grid and \textit{e.t.c.} 

\begin{itemize}
    \item OS: Cross-platform
    \item Lang: C++, Python
    \item Mainly vertex-based dual mesh FVM, also supports FEM and primal grid
\end{itemize}
\end{frame}

\begin{frame}{Vertex-based Dual Grid}
\begin{figure}
    \centering
    \includegraphics[width=0.5\linewidth]{figures/DualMeshControlVolume.png}
    \caption{Dual mesh control volumes surrounding two nodes, $i$ and $j$, in the domain interior.}
    % \label{fig:my_label}
\end{figure}


By virtue of the functionality of geometry-relevant source, researchers do not need to implement the area calculation step and are capable to use the face area and normal vector directly.

\end{frame}

\begin{frame}{CDriver}
CDriver controls instantiating and deleting major classes, just like a housekeeper.\\
\textbf{CDriver::StartSolver} loops over all external iterations to do the following pipeline
\begin{itemize}
    \item PreprocessExtIter
    \item DynamicMeshUpdate (optional)
    \item Run (SIngle ExtIteration)
    \item Update
    \item Monitor
    \item Output
\end{itemize}
\end{frame}

\begin{frame}{CDriver::Run }
CDriver::StartSolver method runs \emph{ExtIter} times the aforementioned pipeline. For steady and unsteady simulation, the procedure is similar, \textbf{steady} solution tends to be stable in the end, while for the \textbf{unsteady} simulation every ExtIter loop means an certain period of physical time development. 
\begin{itemize}
    \item \textbf{Preprocess}\\
            iteration\_container[iZone][Inst\_0]$\longrightarrow$ Preprocess
    \item \textbf{Zone Interface}\\
            interpolator\_container[iZone][jZone]$\longrightarrow$ Set\_TransferCoeff
            Transfer\_Data(iZone, jZone)\\
    \item \textbf{Iterate}\\
            iteration\_container[iZone][Inst\_0]$\longrightarrow$ Iterate
    \item \textbf{Check Convergence}
\end{itemize}
What does the \textbf{Iteration::Iterate} method do?
\end{frame}

\begin{frame}{CIteration::Iterate}
The CIteration::Iterate loops over all the solver to iterate in all equations.
\begin{itemize}
    \item \textbf{Update Global Parameters}\\
    config[val\_iZone]$\longrightarrow$ setGlobalParams
    \item \textbf{Solve the Equation}\\
    integration[val\_iZone][val\_iInst]...\\
    ...[FLOW\_SOL]$\longrightarrow$MultiGrid\_Iteration\\
    ...[TURB\_SOL]$\longrightarrow$SingleGrid\_Iteration\\
    ...[TRAN\_SOL]$\longrightarrow$SingleGrid\_Iteration\\
    ...[HEAT\_SOL]$\longrightarrow$SingleGrid\_Iteration
    \item \textbf{Mesh Update}
    \item \textbf{Output}
    
\end{itemize}
Now our focus moves from \emph{CIteration} class to \emph{CIntegration} class.
\end{frame}

\begin{frame}{CIntegration::Iteration}
Most simple case is CIntegration::SingleGrid\_Iteration.

\begin{itemize}
    \item \textbf{Preprocessing}
    solver\_container[iZone][iInst][FinestMesh][SolContainer\_Position]\\$\longrightarrow$ Preprocessing\\$\longrightarrow$ Set\_OldSolution \\$\longrightarrow$ Set\_Timestep
    \item \textbf{Space Integration}
    Space\_Integration(...)
    \item \textbf{Time Integration}
    Time\_Integration(...)
    \item \textbf{Postprocessing}
\end{itemize}

The space integration will invoke \textbf{CSolver} class to compute the residual.
\end{frame}

\begin{frame}
\frametitle{CSolver}
One physical problem may consist of various equations.
CSolver controls its own conservative equation. 

e.g., RANS model in SU2 has three solvers working together, FLOW\_SOL, TURB\_SOL, TRAN\_SOL.



Let's have heat equation as an example.
Different spatial residual terms coexist in one equation.
\begin{itemize}
    \item \emph{CONV}, Convective terms caused by fluid motion.
    \item \emph{VISC}, Viscous terms caused by the gradient of temperature. 
\end{itemize}

\textbf{Space\_Integration}
solver\_container[MainSolver]...\\
\textbf{CONV\_TERM}$\longrightarrow$Centered\_Residual\quad$\longrightarrow$Upwind\_Residual\\
\textbf{VISC\_TERM}$\longrightarrow$Viscous\_Residual\\
\textbf{Source\_TERM}$\longrightarrow$Source\_Residual\\
\textbf{BC}$\longrightarrow$BC\_Euler\_wall(...)...\\
\textbf{Dual\_Time}$\longrightarrow$SetResidual\_DualTime

\end{frame}


\begin{frame}{CHeatSolver VISC TERM}
Apparently, the viscous term in heat equation means the temperature tends to be the same all over the domain.

For simple situation in solid material, the heat equation reads as
\begin{equation}
    \frac{\partial T}{\partial t} - \nabla\cdot(\alpha \nabla T) =0,
\end{equation}
where $\alpha$ is the thermal diffusivity and depends on whether the material is liquid or solid.

\begin{equation}
F_{ij}=\alpha \vec{S}\cdot \nabla T=\alpha \vec{S}\cdot \frac{T_j-T_I}{\left \| d_{ij} \right \|^2}\vec{d_{ij}}
\end{equation}
\begin{equation}
    \frac{\partial F_{ij}}{\partial T_i}=-\alpha \frac{\vec{S}\cdot \vec{d_{ij}}}{\left \| d_{ij} \right \|^2}
\end{equation}
BCs are similar, contributing residuals and Jacobian matrix to the overall LARGE SPARSE matrix.
\end{frame}


\lstset{keywordstyle=\color{red},keywords={court}}
\begin{lstlisting}
  eddy_viscosity_i = 0.0;
  eddy_viscosity_j = 0.0;
  laminar_viscosity = config->GetMu_ConstantND();
  Prandtl_Lam = config->GetPrandtl_Lam();
  Prandtl_Turb = config->GetPrandtl_Turb();

  for (iEdge = 0; iEdge < geometry->GetnEdge(); iEdge++) {

    iPoint = geometry->edge[iEdge]->GetNode(0);
    jPoint = geometry->edge[iEdge]->GetNode(1);

    /*--- Points coordinates, and normal vector ---*/

    numerics->SetCoord(geometry->node[iPoint]->GetCoord(),
                       geometry->node[jPoint]->GetCoord());
    numerics->SetNormal(geometry->edge[iEdge]->GetNormal());

    Temp_i_Grad = node[iPoint]->GetGradient();
    Temp_j_Grad = node[jPoint]->GetGradient();
    numerics->SetConsVarGradient(Temp_i_Grad, Temp_j_Grad);

    /*--- Primitive variables w/o reconstruction ---*/
    Temp_i = node[iPoint]->GetSolution(0);
    Temp_j = node[jPoint]->GetSolution(0);
    numerics->SetTemperature(Temp_i, Temp_j);

    /*--- Eddy viscosity to compute thermal conductivity ---*/
    if (flow) {
      if (turb) {
        eddy_viscosity_i = solver_container[TURB_SOL]->node[iPoint]->GetmuT();
        eddy_viscosity_j = solver_container[TURB_SOL]->node[jPoint]->GetmuT();
      }
      thermal_diffusivity_i = (laminar_viscosity/Prandtl_Lam) + (eddy_viscosity_i/Prandtl_Turb);
      thermal_diffusivity_j = (laminar_viscosity/Prandtl_Lam) + (eddy_viscosity_j/Prandtl_Turb);
    }
    else {
      thermal_diffusivity_i = config->GetThermalDiffusivity_Solid();
      thermal_diffusivity_j = config->GetThermalDiffusivity_Solid();
    }

    numerics->SetThermalDiffusivity(thermal_diffusivity_i,thermal_diffusivity_j);

    /*--- Compute residual, and Jacobians ---*/

    numerics->ComputeResidual(Residual, Jacobian_i, Jacobian_j, config);

    /*--- Add and subtract residual, and update Jacobians ---*/

    LinSysRes.SubtractBlock(iPoint, Residual);
    LinSysRes.AddBlock(jPoint, Residual);

    Jacobian.SubtractBlock(iPoint, iPoint, Jacobian_i);
    Jacobian.SubtractBlock(iPoint, jPoint, Jacobian_j);
    Jacobian.AddBlock(jPoint, iPoint, Jacobian_i);
    Jacobian.AddBlock(jPoint, jPoint, Jacobian_j);
\end{lstlisting}



% \section{Physics of SU2}

\begin{frame}{Conservative Equations}
\begin{equation}
\frac{\partial \boldsymbol{U}}{\partial t}+\nabla \cdot \boldsymbol{F}^{c}-\nabla \cdot\left(\mu^{v k} \boldsymbol{F}^{v k}\right)=\boldsymbol{Q} \quad \text { in }\Omega,  \quad t>0
\end{equation}
    
\end{frame}

\subsection{Fluid Model}
\begin{frame}[allowframebreaks]
\frametitle{Reynolds-Averaged Navier–Stokes Equations (RANS)}
\begin{equation}
\left\{\begin{array}{ll}{\mathcal{R}(U)=\frac{\partial U}{\partial t}+\nabla \cdot \boldsymbol{F}^{c}-\nabla \cdot\left(\mu^{v k} \boldsymbol{F}^{v k}\right)-\boldsymbol{Q}=0} & {\text { in } \Omega, \quad t>0} \\ {v=\mathbf{0}} & {\text { on }S,} \\ {\partial_{n} T=0} & {\text { on } S} \\ {(\boldsymbol{W})_{+}=W_{\infty}} & {\text { on } \Gamma_{\infty}}\end{array}\right.
\end{equation}


\begin{equation*}
\frac{\partial \boldsymbol{U}}{\partial t}+\nabla \cdot \boldsymbol{F}^{c}-\nabla \cdot\left(\mu^{v k} \boldsymbol{F}^{v k}\right)=Q \quad \text { in } \Omega, t>0
\end{equation*}

\framebreak

\begin{equation*}
\boldsymbol{U}=\left\{\begin{array}{c}{\rho } \\ {\rho \boldsymbol{v} } \\ {\rho E}\end{array}\right\},\quad
\boldsymbol{F}^{c}=\left\{\begin{array}{c}{\rho \boldsymbol{v}} \\ {\rho \boldsymbol{v} \otimes \boldsymbol{v}+ p\mathcal{I} } \\ {\rho E \boldsymbol{v}+p \boldsymbol{v}}\end{array}\right\}
\end{equation*}

\begin{equation*}
\boldsymbol{F}^{v1}=\left\{\begin{array}{c}{\cdot} \\ {\mathcal{\tau}} \\ {\mathcal{\tau} \cdot \boldsymbol{v}}\end{array}\right\},\quad
\boldsymbol{F}^{v2}=\left\{\begin{array}{c}{\cdot} \\ {\cdot} \\ {c_p \nabla T}\end{array}\right\},\quad
Q=\left\{\begin{array}{c}{q_\rho} \\ {\boldsymbol{q}_{\rho\boldsymbol{v}}} \\ {q_{\rho E}}\end{array}\right\}
\end{equation*}

\begin{equation}
\mathcal{\tau}=\nabla \boldsymbol{v}+\nabla \boldsymbol{v}^{T}-\frac{2}{3} \mathcal{I}(\nabla \cdot \boldsymbol{v})
\end{equation}

For \textbf{ideal gas},
\begin{equation}
    p=(\gamma-1)\rho[E-\frac{1}{2}(\boldsymbol{v}\cdot \boldsymbol{v})],
\end{equation}
\begin{equation}
    T=p/(\rho R), 
\end{equation}
\begin{equation}
    c_p = \gamma R/(\gamma-1)
\end{equation}

The total viscosity is divided into a laminar $\mu_{dyn}$  and a
turbulent $\mu_{tur}$ component.
\begin{equation}
\mu^{v 1}=\mu_{\mathrm{dyn}}+\mu_{\mathrm{tur}}, \quad \mu^{v 2}=\frac{\mu_{\mathrm{dyn}}}{P r_{d}}+\frac{\mu_{\mathrm{tur}}}{P r_{t}},
\end{equation}
where $Pr_d$ and $Pr_t$ are the dynamic and turbulent Prandtl numbers,
respectively.



\framebreak
\textbf{Supported Boundary Conditions}
\begin{itemize}
    \item Euler (flow tangency) and symmetry wall,
    \item no-slip wall (adiabatic and isothermal),
    \item far-field and near-field
boundaries,
    \item characteristic-based inlet boundaries (stagnation, mass
flow, or supersonic conditions prescribed),
    \item characteristic-based
outlet boundaries (back pressure prescribed),
    \item periodic boundaries,
    \item nacelle inflow boundaries (fan face Mach number prescribed),
    \item nacelle exhaust boundaries (total nozzle temp and total nozzle
pressure prescribed).
\end{itemize}

The turbulent viscosity $\mu_{tur}$ is obtained from a suitable turbulence
model involving the flow state and a set of new variables. The (1) \textbf{Menter
shear-stress transport} model and the (2) \textbf{Spalart–Allmaras} model are two
of the most common and widely used turbulence models for the
analysis and design of engineering applications in turbulent flows.
    

\end{frame}

\begin{frame}{SU2 Framework}
\uncover<+->{\begin{equation*}
\frac{\partial U}{\partial t}+\nabla \cdot \boldsymbol{F}^{c}-\nabla \cdot\left(\mu^{v k} \boldsymbol{F}^{v k}\right)=Q \quad \text { in } \Omega, t>0
\end{equation*}
}
\uncover<+->{\[ \Downarrow \]
\begin{align*}
    \int_{\Omega_{i}} \frac{\partial U}{\partial t} d \Omega & + \sum_{j \in \mathcal{N}(i)}\left(\tilde{F}_{i j}^{c}+\tilde{F}_{i j}^{v k}\right) \Delta S_{i j}-Q\left|\Omega_{i}\right| \\ 
=\int_{\Omega_{i}} \frac{\partial U}{\partial t} d \Omega & +  R_{i}(U)=0
\end{align*}
}

For each variable in each cell, there is a residual.
\end{frame}

\begin{frame}{Two-Temperature Model for High-Speed Non-Equilibrium Flow}
    
\end{frame}

\begin{frame}{Heat Equation}
    
\end{frame}

\begin{frame}{Wave Equation}
    
\end{frame}

\begin{frame}{Poisson Equation}
    
\end{frame}

\begin{frame}{Equations of Linear Elasticity}
    
\end{frame}
\section{What's Done and Left?}

\begin{frame}{Tasks}

DONE:
\begin{itemize}
    \item $\CheckedBox$  CMaxwellVariable class
    \item $\CheckedBox$  CMaxwellSolver class except for the BC part
    
    \begin{itemize}
        \item Solver Initialization: Initial variable state setting is quite easy compared to other solver's variables. Setting to zero initially is enough.
        
        \item Spatial Residual: flux-splitting method is categorized to be the source term, while the current density influence on $dE/dt$ is will also be added.
    \end{itemize}
    
    \item $\CheckedBox$  CNumerics class using flux-splitting method to compute residual
    \item $\CheckedBox$  CDriver class
    \item $\CheckedBox$  CIntegration class
\end{itemize}

\end{frame}

\begin{frame}{Tasks}
TODO:
\begin{itemize}
    \item $\Box$  BC part of the Solver class
    \begin{itemize}
        \item Non-reflection condition.
        \item PMR, perfect matched layer.
        \item Planar EM wave as source.
        \includegraphics[width=0.2\linewidth]{figures/PlanarWave.png}
        \item Dielectric material, perfect conductor and perfect magnetized material.
    \end{itemize}
    \item $\Box$ Output function 
    \item $\Box$ Classical Test Scene Result Verification
    \item $\Box$ Thin-Wire model
    
    No similar concepts found in fluid dynamics.
\end{itemize}


\end{frame}


% \begin{frame}{Tasks}
% For the EM Maxwell module in unstructured mesh,
% \begin{itemize}
%     \item \emph{What's done?}
%     \begin{itemize}
%         \item Researched on the current stability analysis of the Maxwell flux instability.
%         \item Implemented the flux-splitting numerical method in SU2.
%     \end{itemize}
%     \item \emph{What's left?} 
%     \begin{itemize}
%         \item To implement the boundary condition flux computation.
%         \item To simulate the thin-witr model, which can be used to simulate the coil in tokamaks.
%         \item To couple different modules together.
%     \end{itemize}
% \end{itemize}


% For the magnetic reconnection simulation,
% \begin{itemize}
%     \item Maxwell module
%     \item Pressure module
%     \item Torus geometry setting
% \end{itemize}
% \end{frame}


% \section{Summer Experience}


\begin{frame}{Conferences and Reports}
 Prof. Na, as the chairman of the Integrated Operation Scenario (IOS) Group (part of ITPA), presented a report about the nonlinear MHD effect on disruption predicted by JOREK code and RNN. 
\begin{figure}
\begin{minipage}[t]{0.5\linewidth}
\centering
\includegraphics[width=\linewidth]{figures/ConferenceSchedule.png}
\caption{Part of Conference Schedule}
% \label{fig:side:a}
\end{minipage}%
\begin{minipage}[t]{0.5\linewidth}
\centering
\includegraphics[width=\linewidth]{figures/conference_photo.jpg}
\caption{Conference Scene}
% \label{fig:side:b}
\end{minipage}
\end{figure}

\end{frame}

\begin{frame}{Last Weekend Hiking}
My last weekend in Seoul, our lab has a hiking to an island near Seoul.
\begin{figure}
\centering
\includegraphics[width=0.9\linewidth]{figures/hiking_sight.jpg}
\end{figure}
\end{frame}

\begin{frame}{Last Weekend Hiking}

\begin{figure}
\centering
\includegraphics[width=0.9\linewidth]{figures/group_photo.jpg}
\end{figure}
\end{frame}






% \input{chapters/example-apps}
% \input{chapters/layers}
% \include{chapters/student-projects}

% \plain{End of Session 1\\See you next week!}

% \include{chapters/wordnet}

% \plain{The End\\Thank you!\\\xiheifont 경청해 주셔서 감사합니다.\Winkey\\Farewell, wish you all enjoy new semester.\\Does this presentation have an appropriate length?}

\begin{frame}{The End}
Thanks for your listening. \\
Thanks for my department's sponsor especially and SNU's beautiful sightseeing.
\begin{figure}
\centering
\includegraphics[width=0.9\linewidth]{figures/sightseeing_chair.jpg}
\end{figure}

    
\end{frame}
\plain{May the sun arise from the east.}
\plain{Backup Slides}
\section{BACKUP: Time Integration Method in SU2}

\subsection{Steady Simulations}


\begin{frame}[allowframebreaks]{Implicit Method, Euler Backward Method}

\begin{equation}
\int_{\Omega_{i}} \frac{\partial \boldsymbol{U}_i}{\partial t} d \Omega+\boldsymbol{R}_{i}(\boldsymbol{U}) \approx\left|\Omega_{i}\right| \frac{\mathrm{d} \boldsymbol{U}_{i}}{\mathrm{d} t}+R_{k}(\boldsymbol{U})=0 \quad \rightarrow \quad \frac{\left|\Omega_{i}^{n}\right|}{\Delta t_{i}^{n}} \Delta \boldsymbol{U}_{i}^{n}=-\boldsymbol{R}_{i}\left(\boldsymbol{U}^{n+1}\right)
\end{equation}
where $\Delta \boldsymbol{U}_{i}^{n}=\boldsymbol{U}_{i}^{n+1}-\boldsymbol{U}_{i}^{n}$ means the variables in the cell $i$, $\boldsymbol{R}_i$ is residuals for those variables. However, the residuals at time $n + 1$ are unknown, and linearization about $t^n$ is
needed:

\begin{multline}
    \boldsymbol{R}_{i}\left(\boldsymbol{U}^{n+1}\right)=\boldsymbol{R}_{i}\left(\boldsymbol{U}^{n}\right)+\frac{\partial \boldsymbol{R}_{i}\left(\boldsymbol{U}^{n}\right)}{\partial t} \Delta t_{i}^{n}+\mathcal{O}\left(\Delta t^{2}\right)\\=\boldsymbol{R}_{i}\left(\boldsymbol{U}^{n}\right)+\sum_{j \in \mathcal{N}(i)} \frac{\partial \boldsymbol{R}_{i}\left(\boldsymbol{U}^{n}\right)}{\partial \boldsymbol{U}_{j}} \Delta \boldsymbol{U}_{j}^{n}+\mathcal{O}\left(\Delta t^{2}\right)
\end{multline}

\begin{equation}
\left(\frac{\left|\Omega_{i}\right|}{\Delta t_{i}^{n}} \delta_{i j}+\frac{\partial \boldsymbol{R}_{i}\left(\boldsymbol{U}^{n}\right)}{\partial \boldsymbol{U}_{j}}\right) \cdot \Delta \boldsymbol{U}_{j}^{n}=-\boldsymbol{R}_{i}\left(\boldsymbol{U}^{n}\right),
\end{equation}

If a flux $\tilde{F}_{i j}$ has a stencil of points
${i,j}$, then contributions are made to the Jacobian at four points:
\begin{equation}
\frac{\partial \boldsymbol{R}}{\partial \boldsymbol{U}} :=\frac{\partial \boldsymbol{R}}{\partial \boldsymbol{U}}+\left[\begin{array}{ccccc}{\ddots} & {} & {} & {} & {} \\ {} & {\frac{\partial \tilde{F}_{i j}}{\partial U_{i}}} & {\cdots} & {\frac{\partial \tilde{F}_{i j}}{\partial U_{j}}} \\ {} & {\vdots} & {\ddots} & {\vdots} \\ {} & {-\frac{\partial \tilde{F}_{i j}}{\partial U_{i}}} & {\cdots} & {-\frac{\partial \tilde{F}_{i j}}{\partial U_{j}}} \\ {} & {} & {} & {\ddots}\end{array}\right]
\end{equation}
The position of these four terms are to be checked.


The matrix is very large and contains $(\text{Cell Num} * \text{Variable Num})^2$ components. If one doesn't use the implicit methods in the time-stepping procedure, then there is no necessity to solve the huge sparse linear algebra problem subtly, because all components of the matrix are allocated at the diagonal line.  

Note that, despite implicit schemes being unconditionally stable in theory, a specific value of $\Delta t_i^n$ is needed
to relax the problem. SU2 uses a local-time-stepping technique to accelerate convergence to a steady state.
Local-time-stepping allows each cell in the mesh to advance at a different local time step. Calculation of the
local time step requires the estimation of the eigenvalues and first-order approximations to the Jacobians at
every node i according to

\begin{equation}
\Delta t_{i}=N_{C F L} \min \left(\frac{\left|\Omega_{i}\right|}{\lambda_{i}^{\operatorname{conv}}}, \frac{\left|\Omega_{i}\right|}{\lambda_{i}^{v i s c}}\right)
\end{equation}

where $N_{CFL}$ is the Courant-Friedrichs-Lewy (CFL) number, $\left|\Omega_{i}\right|$ is the volume of the cell $i$ and $\lambda_{i}^{v i s c}$
is the integrated convective spectral radius computed as

\begin{equation}
\lambda_{i}^{c o n v}=\sum_{j \in \mathcal{N}(i)}\left(\left|\vec{u}_{i j} \cdot \vec{n}_{i j}\right|+c_{i j}\right) \Delta S
\end{equation}

where $\vec{u}_{i j}=\left(\vec{u}_{i}+\vec{u}_{j}\right) / 2$, and $c_{i j}=\left(c_{i}+c_{j}\right) / 2$ denote the velocity and the speed of sound at the cell face.
$\vec{n}_{i j}$ denotes the normal direction of the control surface and $\Delta S$, its area. On the other hand, the viscous
spectral radius $\lambda_{i}^{v i s c}$ is computed as
\begin{equation}
\lambda_{i}^{v i s c}=\sum_{j \in \mathcal{N}(i)} C \frac{\mu_{i j}}{\rho_{i j}} S_{i j}^{2}
\end{equation}
where $C$ is a constant, $\mu_{i j}$ is the sum of the laminar and eddy viscosities in a turbulent calculation and $\rho_{i j}$
is the density evaluated at the midpoint of the edge ${i,j}$.

\end{frame}

\begin{frame}[allowframebreaks]{Dual-Time-Stepping}
\begin{equation}
\frac{\partial \boldsymbol{U}}{\partial \tau}+\boldsymbol{R}_{i}^{*}(\boldsymbol{U})=0
\end{equation}
with
\begin{equation}
\begin{aligned} R_{i}^{*}(\boldsymbol{U})=& \frac{3}{2 \Delta t} \boldsymbol{U}_{i} \\+& \frac{1}{\left|\Omega_{i}\right|^{n+1}}\left(\boldsymbol{R}_{i}(\boldsymbol{U})-\frac{2}{\Delta t}\left|\Omega_{i}\right|^{n} \boldsymbol{U}_{i}^{n}+\frac{1}{2 \Delta t}\left|\Omega_{i}\right|^{n-1} \boldsymbol{U}_{i}^{n-1}\right) \end{aligned}
\end{equation}
for second-order accuracy in time (backward difference formula),
where $\Delta t$ is the physical time step, $\tau$ is a fictitious time used to
converge the steady-state problem, $\boldsymbol{R}_i(\boldsymbol{U})$ denotes the residual of the
governing equations.

\end{frame}
\appendix

\begin{frame}[allowframebreaks]
\frametitle{Bibliography}


\nocite{*}
\printbibliography[heading=none]

\end{frame}


\end{document}